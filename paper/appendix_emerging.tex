%=============================================================================
% APPENDIX A: EMERGING MECHANISMS UNDER INVESTIGATION
%=============================================================================

\clearpage
\appendix

\section{Emerging Mechanisms Under Investigation}
\label{appendix:emerging}

This appendix catalogs computational patterns that show preliminary evidence but do not yet meet the independence criteria (Section~\ref{sec:framework}) for inclusion in the main taxonomy. These patterns may be:

\begin{itemize}[itemsep=0.2em]
\item Emergent behaviors from documented mechanism composition
\item Patterns requiring stronger mechanistic characterization
\item Potentially valid mechanisms needing independent confirmation across models
\item Components of other mechanisms rather than independent computational functions
\end{itemize}

Researchers are encouraged to investigate these patterns and provide evidence for or against their status as independent mechanisms using the four-criterion framework: (1) distinct ablation signature, (2) separable implementation, (3) independent activation, (4) non-compositional nature.

%-----------------------------------------------------------------------------
\subsection{Memory Consolidation}
\label{app:memory-consolidation}

\textbf{Hypothesized Function:} Integrate multiple memory sources (facts, schemas, entities) into coherent knowledge representations for complex queries requiring synthesis across memory types.

\textbf{Current Evidence:}
\begin{itemize}[itemsep=0.1em]
\item Behavioral observations suggest multi-source integration for complex queries
\item Ablation at layer level shows some degradation on knowledge-intensive tasks
\item Queries like ``Describe Einstein's contributions to physics in the context of early 20th century science'' appear to require coordinated retrieval
\end{itemize}

\textbf{Independence Challenge - Fails Criterion 4 (Non-Compositional):}

This pattern appears to be emergent composition of three documented mechanisms:
\begin{itemize}[itemsep=0.1em]
\item \textbf{Factual Recall} (Section~\ref{mech:factual-recall}): Retrieves Einstein facts
\item \textbf{Schema Retrieval} (Section~\ref{mech:schema-retrieval}): Accesses ``early 20th century science'' temporal/contextual framework
\item \textbf{Entity Grounding} (Section~\ref{mech:entity-grounding}): Links Einstein entity across mentions
\end{itemize}

No unique implementing structure has been identified beyond the coordination of these three mechanisms. The ``integration'' appears to occur through standard attention-MLP composition rather than specialized circuitry.

\textbf{Research Needed:}
\begin{itemize}[itemsep=0.1em]
\item Demonstrate ablation effects not explained by ablating constituent mechanisms
\item Identify unique implementing components (attention heads or MLP neurons) not shared with Factual Recall, Schema Retrieval, or Entity Grounding
\item Show independent activation: can Memory Consolidation be suppressed while preserving the three constituent mechanisms?
\item Alternative hypothesis: This may represent general compositional capability rather than a specific mechanism
\end{itemize}

%-----------------------------------------------------------------------------
\subsection{Local Context Modeling}
\label{app:local-context-modeling}

\textbf{Hypothesized Function:} Process immediate local context (1--5 tokens) for instruction markers, syntactic structure, and local patterns. Distinct from Pattern Completion which operates at broader scale (10+ tokens).

\textbf{Current Evidence:}
\begin{itemize}[itemsep=0.1em]
\item Behavioral observations of instruction detection in early layers
\item Format-sensitive tasks appear to rely on early-layer processing
\item N-gram neurons encode local sequential patterns
\end{itemize}

\textbf{Independence Challenge - May Be Component of Pattern Completion:}

Several concerns about independence:
\begin{enumerate}[itemsep=0.1em]
\item \textbf{Scale distinction unclear:} Is processing 1--5 tokens fundamentally different from 10+ tokens, or just Pattern Completion at smaller scale?
\item \textbf{Shared implementation:} Uses same components (n-gram neurons, early attention heads) as Pattern Completion
\item \textbf{Separability unclear:} Cannot ablate ``local'' pattern detection without affecting broader pattern detection
\item \textbf{May be implementation detail:} Early-layer operation may be \textit{how} Pattern Completion works rather than a separate mechanism
\end{enumerate}

\textbf{Research Needed:}
\begin{itemize}[itemsep=0.1em]
\item Demonstrate that local context processing (1--5 tokens) can be ablated independently of broader pattern completion (10+ tokens)
\item Identify components that respond specifically to local context without participating in broader pattern detection
\item Show qualitatively different computation for local vs. broader scales
\item Alternative hypothesis: This may describe the early-layer implementation of Pattern Completion rather than an independent mechanism
\end{itemize}

%-----------------------------------------------------------------------------
\subsection{Structural Boundary Tracking}
\label{app:structural-boundary-tracking}

\textbf{Hypothesized Function:} Track structural boundaries (sentence, paragraph, section markers, formatting delimiters, nested structure levels) to enable structure-aware processing.

\textbf{Current Evidence:}
\begin{itemize}[itemsep=0.1em]
\item Weak behavioral evidence from format-sensitive tasks
\item Some attention heads appear to attend to delimiter tokens
\item Models show some sensitivity to document structure
\end{itemize}

\textbf{Independence Challenge - Insufficient Evidence and Potential Component Status:}

Multiple concerns:
\begin{enumerate}[itemsep=0.1em]
\item \textbf{Insufficient mechanistic evidence:} Primarily inferred from behavioral observations without clear component identification
\item \textbf{May be Position-Based Processing:} Structural boundaries may be tracked through positional encodings rather than specialized mechanism
\item \textbf{May be Format Enforcement component:} Boundary awareness may be an implementation detail of how Format Enforcement works
\item \textbf{Weak ablation signature:} No clear demonstration of what behaviors are uniquely lost when ``boundary tracking'' is removed
\end{enumerate}

\textbf{Research Needed:}
\begin{itemize}[itemsep=0.1em]
\item Identify specific components (attention heads or MLP neurons) dedicated to boundary detection
\item Demonstrate ablation effects distinct from Position-Based Processing and Format Enforcement
\item Show structure tracking operates independently of positional information
\item Clarify whether this describes a mechanism or an implementation detail of other mechanisms
\item Alternative hypothesis: This may be an aspect of Position-Based Processing applied to structural markers
\end{itemize}

%-----------------------------------------------------------------------------
\subsection{Repetition \& Cycle Recognition}
\label{app:repetition-cycle-recognition}

\textbf{Note:} This pattern was merged into Pattern Completion Mechanism (Section~\ref{mech:pattern-completion}) as an implementation detail rather than relegated to this appendix. We include brief notes here for completeness.

\textbf{Status:} Repetition detection via duplicate-token heads serves multiple downstream mechanisms (pattern completion, entity tracking, output routing). However, this multi-purpose functionality represents component reuse rather than an independent mechanism. Duplicate-token heads are now documented as part of the Pattern Completion circuit implementation.

\textbf{Rationale for merger:} Repetition detection doesn't meet Criterion 2 (Separable Implementation) - the duplicate-token heads are shared across Pattern Completion, Entity Grounding, and Output Routing. This is computational reuse of components rather than an independent mechanism.

%-----------------------------------------------------------------------------
\subsection{Guidelines for Future Mechanism Proposals}

When proposing new mechanisms, researchers should:

\textbf{1. Apply Independence Criteria:}
\begin{itemize}[itemsep=0.1em]
\item Demonstrate distinct ablation signature not explained by other mechanisms
\item Identify separable implementing components with some unique structure
\item Show independent activation through targeted interventions
\item Prove non-compositional: not just emergent behavior from documented mechanisms
\end{itemize}

\textbf{2. Provide Strong Evidence:}
\begin{itemize}[itemsep=0.1em]
\item Systematic ablation studies across multiple models
\item Component-level identification (specific attention heads, MLP neurons, or circuits)
\item Reproducible demonstrations using standard techniques
\item Clear behavioral predictions that distinguish from similar mechanisms
\end{itemize}

\textbf{3. Consider Alternative Explanations:}
\begin{itemize}[itemsep=0.1em]
\item Could this be component reuse rather than a new mechanism?
\item Could this be composition of existing mechanisms?
\item Could this be an implementation detail of a broader mechanism?
\item Could this be explained by architectural primitives (position encoding, attention masking)?
\end{itemize}

\textbf{4. Document Limitations:}
\begin{itemize}[itemsep=0.1em]
\item Acknowledge evidence gaps and alternative interpretations
\item Specify which independence criteria are well-supported vs. uncertain
\item Identify research needed to definitively establish independence
\end{itemize}

The goal is cumulative mechanistic knowledge: each validated mechanism should advance understanding rather than multiply terminology for the same computational functions.
