\documentclass[11pt,a4paper]{article}

% Load preamble with packages and macros
%=============================================================================
% PREAMBLE.TEX - Packages, Macros, and Formatting
% For: Attention Heads, MLPs, and Circuits: A Naming Convention for 
%      Transformer Mechanisms
%=============================================================================

%-----------------------------------------------------------------------------
% Core Packages
%-----------------------------------------------------------------------------
\usepackage[utf8]{inputenc}
\usepackage[T1]{fontenc}

%-----------------------------------------------------------------------------
% Page Layout
%-----------------------------------------------------------------------------
\usepackage[margin=1in]{geometry}
\usepackage{setspace}
\onehalfspacing

%-----------------------------------------------------------------------------
% Mathematics
%-----------------------------------------------------------------------------
\usepackage{amsmath}
\usepackage{amssymb}
\usepackage{amsthm}

%-----------------------------------------------------------------------------
% Tables and Figures
%-----------------------------------------------------------------------------
\usepackage{booktabs}
\usepackage{longtable}
\usepackage{array}
\usepackage{tabularx}
\usepackage{graphicx}
\usepackage{float}
\usepackage{svg}

%-----------------------------------------------------------------------------
% Colors and Styling
%-----------------------------------------------------------------------------
\usepackage{xcolor}
\definecolor{darkblue}{rgb}{0.0, 0.0, 0.5}
\definecolor{darkgreen}{rgb}{0.0, 0.5, 0.0}
\definecolor{darkred}{rgb}{0.5, 0.0, 0.0}

%-----------------------------------------------------------------------------
% Hyperlinks and References
%-----------------------------------------------------------------------------
\usepackage{hyperref}
\hypersetup{
    colorlinks=true,
    linkcolor=darkblue,
    citecolor=darkgreen,
    urlcolor=darkblue,
    bookmarksnumbered=true,
    pdfborder={0 0 0}
}
\usepackage{cleveref}

%-----------------------------------------------------------------------------
% Lists and Enumeration
%-----------------------------------------------------------------------------
\usepackage{enumitem}
\setlist{itemsep=0.1em, parsep=0.1em, topsep=0.3em}

%-----------------------------------------------------------------------------
% Boxes and Styling - Load tcolorbox package FIRST
%-----------------------------------------------------------------------------
\usepackage{tcolorbox}
\tcbuselibrary{skins,breakable}

%-----------------------------------------------------------------------------
% Section Formatting
%-----------------------------------------------------------------------------
\usepackage{titlesec}
\titlespacing*{\section}{0pt}{1.5em}{0.8em}
\titlespacing*{\subsection}{0pt}{1.0em}{0.5em}
\titlespacing*{\subsubsection}{0pt}{1em}{0.3em}

%-----------------------------------------------------------------------------
% Table of Contents
%-----------------------------------------------------------------------------
\setcounter{tocdepth}{3}
\setcounter{secnumdepth}{3}
\usepackage{tocloft}
\setlength{\cftbeforesecskip}{2pt}
\setlength{\cftbeforesubsecskip}{1pt}
\setlength{\cftbeforesubsubsecskip}{1pt}

%-----------------------------------------------------------------------------
% Bibliography
%-----------------------------------------------------------------------------
\usepackage{natbib}
\setcitestyle{square,numbers,comma}

%-----------------------------------------------------------------------------
% Paragraph Formatting
%-----------------------------------------------------------------------------
\setlength{\parindent}{0pt}
\setlength{\parskip}{0.5em}

%-----------------------------------------------------------------------------
% Typography Enhancements
%-----------------------------------------------------------------------------
\usepackage[protrusion=true,expansion=false]{microtype}
\usepackage{csquotes}
\usepackage{xspace}

%=============================================================================
% CUSTOM COMMANDS AND ENVIRONMENTS FOR MECHANISM ENTRIES
%=============================================================================

%-----------------------------------------------------------------------------
% Depth Indicators
%-----------------------------------------------------------------------------
\newcommand{\Early}{\textsf{E}}
\newcommand{\Middle}{\textsf{M}}
\newcommand{\Late}{\textsf{L}}
\newcommand{\Final}{\textsf{F}}

%-----------------------------------------------------------------------------
% Mechanism Entry Formatting
%-----------------------------------------------------------------------------
\newcommand{\depthinfo}[1]{\textbf{Depth:} \texttt{#1}}
\newcommand{\primaryimpl}[1]{\textbf{Primary Implementation:} #1}
\newcommand{\litnames}[1]{\textbf{Literature names:} \textit{#1}}
\newcommand{\statuswell}{\textsc{Tier-1: Mechanistic}}
\newcommand{\statusobs}{\textsc{Tier-2: Behavioral}}
\newcommand{\statusprop}{\textsc{Tier-3: Speculative}}
\newcommand{\statustierone}{\textsc{Tier-1: Mechanistic}}
\newcommand{\statustiertwo}{\textsc{Tier-2: Behavioral}}
\newcommand{\relatedmechs}[1]{\textbf{Related:} #1}
\newcommand{\statusinfo}[1]{\textbf{Status:} #1}

% Compact footer: Status | Related
\newcommand{\mechfooter}[2]{\noindent\statusinfo{#1} | \relatedmechs{#2}}

% Head-specific footer
\newcommand{\headfooter}[2]{\noindent\statusinfo{#1} | \relatedmechs{#2}}

%-----------------------------------------------------------------------------
% Implementation Pattern Formatting
%-----------------------------------------------------------------------------
\newcommand{\attnimpl}[1]{\textbf{Attention Heads:} #1}
\newcommand{\mlpimpl}[1]{\textbf{MLP Layers:} #1}
\newcommand{\circuitimpl}[1]{\textbf{Circuit:} #1}
\newcommand{\saeimpl}[1]{\textbf{SAE Features:} #1}
\newcommand{\archimpl}[1]{\textbf{Architecture:} #1}

%-----------------------------------------------------------------------------
% Example Scenario Formatting
%-----------------------------------------------------------------------------
\newcommand{\exinput}[1]{\textit{Input:} #1}
\newcommand{\exmechanism}[1]{\textit{Mechanism:} #1}
\newcommand{\exeffect}[1]{\textit{Effect:} #1}
\newcommand{\exbehavior}[1]{\textit{Behavior:} #1}

% Attention Pattern Formatting
\newcommand{\attstrong}[1]{\textbf{Strong:} #1}
\newcommand{\attweak}[1]{\textbf{Weak:} #1}
\newcommand{\attreacts}[1]{\textbf{Reacts to:} #1}

%-----------------------------------------------------------------------------
% Style Guide Compliance Helpers
%-----------------------------------------------------------------------------
\newcommand{\eg}{e.g.,\xspace}
\newcommand{\ie}{i.e.,\xspace}
\newcommand{\etc}{etc.\xspace}

%-----------------------------------------------------------------------------
% Custom Environments for Mechanism Entries
%-----------------------------------------------------------------------------

% Function Description
\newenvironment{functiondesc}{%
    \par\vspace{0.3em}%
    \noindent%
}{\par\vspace{0.3em}}

% Mechanism Description
\newenvironment{mechanismdesc}{%
    \par\vspace{0.3em}%
    \noindent%
}{\par\vspace{0.3em}}

% Attention Pattern Box
\newtcolorbox{attentionbox}{
    colback=blue!3,
    colframe=blue!30,
    boxrule=0.5pt,
    arc=2pt,
    left=3pt,
    right=3pt,
    top=2pt,
    bottom=2pt,
    boxsep=0pt,
    before skip=0.5em,
    after skip=0.5em
}

% Implementation Box
\newtcolorbox{implementationbox}{
    colback=gray!5,
    colframe=gray!40,
    boxrule=0.5pt,
    arc=2pt,
    left=3pt,
    right=3pt,
    top=2pt,
    bottom=2pt,
    boxsep=0pt,
    before skip=0.5em,
    after skip=0.5em
}

% Example Box
\newtcolorbox{examplebox}{
    colback=gray!8,
    colframe=gray!50,
    boxrule=0.5pt,
    arc=2pt,
    left=4pt,
    right=4pt,
    top=3pt,
    bottom=3pt,
    boxsep=0pt,
    before skip=0.5em,
    after skip=0.5em,
    fonttitle=\small\bfseries,
    title=Example
}

% Ablation Box
\newtcolorbox{ablationbox}{
    colback=white,
    colframe=gray!30,
    boxrule=0.5pt,
    arc=2pt,
    left=3pt,
    right=3pt,
    top=2pt,
    bottom=2pt,
    boxsep=0pt,
    before skip=0.5em,
    after skip=0.5em
}

% Open Questions Box
\newtcolorbox{openquestionsbox}{
    colback=yellow!5,
    colframe=orange!40,
    boxrule=0.5pt,
    arc=2pt,
    left=4pt,
    right=4pt,
    top=3pt,
    bottom=3pt,
    boxsep=0pt,
    before skip=0.5em,
    after skip=0.5em,
    fonttitle=\small\bfseries,
    title=Open Questions
}

%-----------------------------------------------------------------------------
% Quality Assurance Comments (remove before final compilation)
%-----------------------------------------------------------------------------
\newcommand{\TODO}[1]{\textcolor{red}{\textbf{TODO:} #1}}
\newcommand{\FIXME}[1]{\textcolor{red}{\textbf{FIXME:} #1}}
\newcommand{\CHECKSTYLE}[1]{\textcolor{blue}{\textbf{STYLE:} #1}}

%-----------------------------------------------------------------------------
% End of Preamble
%-----------------------------------------------------------------------------


% Document metadata
\title{Transformer Mechanisms: A Functional Taxonomy}
\author{Karol Kowalczyk}
\date{November 2025}

\begin{document}

\maketitle

\begin{abstract}
Transformer models implement computation through mechanisms that span attention heads, MLP layers, and multi-component circuits. Current mechanistic interpretability research suffers from fragmented naming: the same mechanism appears under different names across research traditions, while identical names refer to distinct computational functions. This work proposes a mechanism-first taxonomy organizing 35 computational mechanisms into 8 functional stacks, with each mechanism described independently of its implementation substrate. The taxonomy enables cross-architecture comparison, supports multi-component integration, and provides stable vocabulary synthesizing findings from attention analysis, MLP studies, circuit tracing, and sparse autoencoder research.
\end{abstract}

\tableofcontents
\clearpage

%=============================================================================
% INTRODUCTION
%=============================================================================

\section{Introduction}
\label{sec:introduction}

\subsection{The Problem: Fragmented Mechanism Naming}

Mechanistic interpretability has identified induction heads enabling few-shot learning~\cite{olsson2022context}, characterized MLPs as key-value memories~\cite{geva2021transformer}, traced multi-component circuits like the IOI circuit~\cite{wang2022interpretability}, and extracted monosemantic features via sparse autoencoders~\cite{bricken2023towards}. Yet naming conventions remain inconsistent and component-centric.

The same computational mechanism appears under multiple names: pattern completion is called "induction head," "pattern head," "copy head," or "ICL head" depending on research tradition. Conversely, "copy head" refers to at least three distinct mechanisms: pattern completion, name-mover routing, and position-based copying.

The root problem is \textbf{component-first thinking}: organizing by implementation substrate (attention heads, MLP neurons) rather than computational function. This creates artificial boundaries preventing systematic comparison across models and obscuring relationships between components and the mechanisms they implement.

\subsection{The Solution: Mechanism-First Organization}

This taxonomy applies three organizing principles:

\textbf{1. Separate mechanism from implementation.} Each mechanism describes computational transformation independently, then maps to implementing components (attention heads, MLPs, circuits, SAE features).

\textbf{2. Organize by computational function.} Eight functional stacks reflect the transformer pipeline: pattern detection, memory retrieval, routing, transformation, reasoning, safety, output control, and composition.

\textbf{3. Support multi-component integration.} Sophisticated mechanisms require coordinated computation across attention, MLPs, and depth. The taxonomy explicitly represents this composition.

\subsection{Scope and Contributions}

This work catalogs 35 mechanisms meeting four criteria: (1) observed across multiple models, (2) validated through ablation, (3) mechanistically understood at component level, (4) reproducible using standard techniques.

\textbf{Key contributions:}
\begin{itemize}[itemsep=0.2em]
\item Standardized vocabulary bridging attention, MLP, circuit, and SAE research traditions
\item Functional organization revealing transformer computational architecture
\item Multi-component integration patterns for sophisticated mechanisms
\item Implementation specifications enabling cross-architecture comparison
\item Empirically grounded descriptions synthesizing interpretability findings
\end{itemize}

\subsection{Document Structure}

Section~\ref{sec:framework} explains the mechanism-first framework and depth-based organization. Sections~\ref{sec:pattern-stack} through~\ref{sec:composition-stack} catalog mechanisms by functional stack. Section~\ref{sec:discussion} examines patterns, limitations, and future directions.

%=============================================================================
% FRAMEWORK
%=============================================================================

\section{Conceptual Framework}
\label{sec:framework}

\subsection{Mechanism vs. Component}

\textbf{Mechanisms} describe computational transformations independently of substrate: pattern completion, factual recall, output routing. \textbf{Components} are architectural primitives: attention heads, MLP layers, circuits. The relationship is many-to-many: mechanisms typically require multiple components, and components contribute to multiple mechanisms.

\textbf{Three organizational levels:}
\begin{enumerate}[itemsep=0.2em]
\item \textbf{Attention heads} implement learned routing patterns (previous-token, induction, name-mover)
\item \textbf{MLP layers} implement associative memory and nonlinear transformation
\item \textbf{Circuits} integrate attention and MLPs across layers into complex behaviors
\end{enumerate}

\subsection{Depth-Based Organization}

Transformer layers exhibit systematic functional specialization across depth. This taxonomy uses relative depth notation: $d_{\text{rel}} = l/L$ where $l$ is layer index and $L$ is total layers.

\textbf{Four depth regions:}
\begin{itemize}[itemsep=0.2em]
\item \textbf{Early (E, 0.00--0.25):} Surface processing, syntactic patterns, instruction markers
\item \textbf{Middle (M, 0.25--0.70):} Core computation, facts, entities, reasoning
\item \textbf{Late (L, 0.70--0.88):} Integration, output routing, strategy selection
\item \textbf{Final (F, 0.88--1.00):} Constraint enforcement, safety, format control
\end{itemize}

This enables architecture-independent mechanism descriptions: a mechanism at 0.40 relative depth occupies similar functional space in 12-layer (layer 5), 48-layer (layer 19), and 96-layer (layer 38) models.

\subsection{Methodology and Evidence Standards}

Mechanisms are identified through converging evidence from five techniques: attention pattern analysis, ablation studies, activation patching, logit attribution, and sparse autoencoder analysis.

All mechanisms in this taxonomy are validated through systematic empirical investigation. However, the depth of mechanistic understanding varies significantly. To address this, we employ a tier system reflecting evidence quality and a formal independence framework.

\subsection{Mechanism Independence Criteria}

A computational pattern qualifies as an independent mechanism if it meets \textbf{all four criteria}:

\textbf{1. Distinct Ablation Signature:} Removal produces unique behavioral deficits not fully explained by ablating other mechanisms. The mechanism must have identifiable functional consequences that distinguish it from related mechanisms.

\textit{Example:} Removing induction heads specifically impairs few-shot learning while preserving other pattern-matching capabilities. This distinct signature establishes independence from general pattern detection.

\textbf{2. Separable Implementation:} Can be localized to specific components or circuits that are not completely shared with other mechanisms. The mechanism may share some components but must have unique implementing structure.

\textit{Example:} The IOI circuit shares duplicate-token heads with pattern completion but has unique S-inhibition and name-mover coordination, establishing separable implementation.

\textbf{3. Independent Activation:} Can be activated or suppressed independently of related mechanisms through targeted interventions. The mechanism must be controllable without necessarily affecting all related computations.

\textit{Example:} Induction heads can be ablated without removing all duplicate-token functionality, demonstrating independent activation control.

\textbf{4. Non-Compositional:} Not fully explainable as a simple composition of other documented mechanisms. Emergent behaviors from mechanism interaction don't qualify as separate mechanisms unless they develop distinct implementation.

\textit{Counterexample:} Memory consolidation appears to be the emergent composition of factual recall + schema retrieval + entity grounding, lacking unique implementing structure. It fails the non-compositional criterion.

These criteria, informed by causal abstraction frameworks~\cite{geiger2021causal} and rigorous testing methodologies~\cite{chan2022causal}, provide systematic standards for mechanism identification and prevent taxonomic inflation from classifying every behavioral pattern as an independent mechanism.

\subsection{Confidence Tiers}

Mechanisms are categorized by the strength and type of empirical support:

\textbf{Tier 1 - Mechanistic Understanding:}
\begin{itemize}[itemsep=0.1em]
\item Circuit-level understanding with identified implementing components
\item Extensive ablation studies demonstrating causal role across multiple models
\item Reproducible using standard interpretability techniques
\item Clear implementation specifications enabling cross-architecture identification
\item \textit{Examples:} Induction mechanism, IOI circuit, factual recall, position-based processing
\end{itemize}

\textbf{Tier 2 - Behavioral Evidence:}
\begin{itemize}[itemsep=0.1em]
\item Consistent behavioral evidence across models and contexts
\item Partial mechanistic understanding with some component identification
\item Preliminary ablation or intervention studies showing functional role
\item Implementation details partially characterized but incomplete
\item \textit{Examples:} Algorithmic continuation, analogical mapping, causal inference, most reasoning mechanisms
\end{itemize}

\textbf{Tier 3 - Speculative (Appendix Only):}
\begin{itemize}[itemsep=0.1em]
\item Reasonable hypotheses with preliminary behavioral observations
\item Insufficient mechanistic characterization for main taxonomy inclusion
\item May be emergent behaviors from other mechanism composition
\item Require substantial additional research for validation
\item Relocated to Appendix A: "Emerging Mechanisms Under Investigation"
\end{itemize}

\textbf{Current taxonomy composition:} 33 mechanisms in main taxonomy (11 Tier 1, 22 Tier 2), with 3 mechanisms relocated to appendix. This distribution reflects the current state of mechanistic interpretability: strong understanding of core attention-based mechanisms, behavioral characterization of many MLP and reasoning mechanisms, and substantial work remaining to achieve complete mechanistic understanding across all computational functions.

%=============================================================================
% MECHANISM STACKS OVERVIEW
%=============================================================================

\section{Eight Functional Stacks}
\label{sec:stacks-overview}

The taxonomy organizes 33 mechanisms into eight functional stacks:

\textbf{1. Pattern \& Sequential Stack} (3 mechanisms, 0.05--0.65): Pattern detection and completion, sequence continuation, positional processing.

\textbf{2. Memory \& Knowledge Stack} (5 mechanisms, 0.35--0.85): Factual recall, entity grounding, schema retrieval, output routing.

\textbf{3. Routing \& Context Stack} (4 mechanisms, 0.35--0.75): Relevance filtering, attention focusing, task routing, context aggregation.

\textbf{4. Feature Transformation Stack} (3 mechanisms, 0.20--0.80): Nonlinear composition, abstract concept formation, semantic integration.

\textbf{5. Reasoning \& Inference Stack} (5 mechanisms, 0.40--0.88): Multi-step reasoning, planning, consistency checking, analogical mapping, causal inference.

\textbf{6. Safety \& Policy Stack} (5 mechanisms, 0.05--0.98): Harmful content detection, policy enforcement, refusal generation, jailbreak resistance.

\textbf{7. Output \& Quality Stack} (4 mechanisms, 0.35--0.98): Format enforcement, style modulation, explanation generation, completion control.

\textbf{8. Composition \& Integration Stack} (4 mechanisms, system-level): Cross-layer circuits, multi-head coordination, attention-MLP composition, representational superposition.

%=============================================================================
% MECHANISM CATALOG
%=============================================================================

\section{Mechanism Catalog}
\label{sec:catalog}

This section catalogs mechanisms using a compressed format: \textbf{Function} describes computational transformation, \textbf{Implementation} specifies components, \textbf{Ablation} summarizes behavioral changes when removed, \textbf{Example} provides concrete input→mechanism→output.

% Individual stack files
%=============================================================================
\subsection{Pattern \& Sequential Stack}
\label{sec:pattern-stack}

\textbf{Stack overview:} Detect and complete patterns from context. Enable in-context learning through pattern matching and repetition detection. Support sequence continuation and algorithmic behavior. Process positional information for order-aware computation.

\textbf{Note:} Two mechanisms from this stack have been relocated: Local Context Modeling and Repetition \& Cycle Recognition moved to Appendix A (may be components of Pattern Completion rather than independent mechanisms).

%-----------------------------------------------------------------------------
\subsubsection{Pattern Completion Mechanism}
\label{mech:pattern-completion}

\noindent\depthinfo{0.05--0.58 (E-M)} | \primaryimpl{Attention heads + MLP neurons} | \litnames{induction mechanism, pattern completion, in-context learning}

\begin{functiondesc}
Detect and complete patterns of form [A][B]...[A] $\rightarrow$ predict [B]. Enable in-context learning through multi-stage circuit combining position-based copying, repetition detection, pattern matching, and n-gram retrieval. Foundation for few-shot learning and analogical reasoning. Support pattern-based prediction without parameter updates. Implement through coordinated multi-component circuit spanning 3--8 layers.

\textbf{Repetition detection component:} Duplicate-token heads detect repeated elements at multiple scales (token-level, phrasal, cyclical patterns) and provide repetition signals for pattern matching, entity tracking, output routing, and disambiguation. This repetition detection serves multiple downstream mechanisms as an implementation detail of the broader pattern completion circuit.
\end{functiondesc}

\begin{implementationbox}
\attnimpl{Previous-token heads (E, 0.05--0.20) create shifted representations through uniform offset attention. Duplicate-token heads (M, 0.30--0.58) detect repeated elements via exact token matching and provide repetition signals. Induction heads (M, 0.25--0.55) match patterns by attending to previous occurrences of current token content.}\\
\mlpimpl{N-gram neurons (E-M, 0.15--0.55) store frequent bigram and trigram patterns as key-value associations. Provide statistical support for pattern completion.}\\
\circuitimpl{Previous-token $\rightarrow$ Induction $\rightarrow$ MLP retrieval working in composition across 3--8 layers. Unified pattern completion circuit integrating multiple specialized components. Duplicate-token heads serve pattern completion, entity tracking, and output routing (IOI circuit) through multi-purpose repetition detection.}
\end{implementationbox}

\begin{ablationbox}
\textbf{Expected ablation:} Severe loss of in-context learning capability. Major degradation on few-shot tasks requiring pattern-based generalization. Loss of pattern completion and analogical continuation. Reduced ability to learn from examples provided in prompt without fine-tuning. Moderate impact on entity tracking and IOI circuit performance due to loss of repetition signals.
\end{ablationbox}

\begin{examplebox}
\exinput{``When Mary and John went to the store, Mary gave milk to John. When Susan and Bob went to the store, Susan gave milk to...''}\\
\exbehavior{Detect pattern [A] gave milk to [B], previous-token creates shifted representations, induction heads match pattern, duplicate-token tracks repetition}\\
\exeffect{Output ``Bob'' by analogical pattern completion across contexts}

\vspace{0.3em}
\textbf{Repetition detection:} ``The cat climbed the tree. The cat...'' $\rightarrow$ Second ``cat'' detected as duplicate, enabling entity linking and pattern completion.
\end{examplebox}

\mechfooter{\statuswell}{algorithmic-continuation, entity-grounding, output-routing}

%-----------------------------------------------------------------------------
\subsubsection{Algorithmic Continuation Mechanism}
\label{mech:algorithmic-continuation}

\noindent\depthinfo{0.35--0.65 (M)} | \primaryimpl{Attention heads + MLP neurons} | \litnames{sequence continuation, rule following, mathematical pattern completion}

\begin{functiondesc}
Continue algorithmic sequences by detecting underlying rules: counting, arithmetic progressions, alphabetic sequences, mathematical patterns. Extract systematic rules from examples and apply consistently. Support rule-based generation beyond memorization. Enable mathematical and logical pattern completion. Integration point between pattern matching and reasoning mechanisms. Implement structured sequence understanding.
\end{functiondesc}

\begin{implementationbox}
\attnimpl{Algorithmic heads (M, 0.35--0.60) detect regular patterns and systematic relationships in sequences through content-based attention to sequence structure.}\\
\mlpimpl{Rule storage neurons (M, 0.40--0.65) encode mathematical operations and sequence transformation rules. Store procedural knowledge for systematic continuation.}\\
\circuitimpl{Integrates with reasoning mechanisms (M-L) for complex rule application. Bridges pattern detection and symbolic reasoning.}
\end{implementationbox}

\begin{ablationbox}
\textbf{Expected ablation:} Significant loss of algorithmic continuation ability. Major degradation on sequence completion tasks requiring rule extraction. Reduced performance on mathematical and logical patterns. Loss of systematic rule application. Increased reliance on memorized sequences rather than rule understanding.
\end{ablationbox}

\begin{examplebox}
\exinput{``2, 4, 8, 16, 32, ...''}\\
\exbehavior{Detect doubling pattern through ratio analysis, activate multiplication rule, apply systematically}\\
\exeffect{Output ``64'' as next power of 2 through rule-based continuation}
\end{examplebox}

\mechfooter{\statusobs}{pattern-completion, multi-step-reasoning}

\begin{openquestionsbox}
\textbf{Mechanistic characterization needed:} While behavioral evidence is strong, detailed circuit-level tracing of rule extraction and application mechanisms remains incomplete. Future research should identify specific neuron populations encoding different rule types and trace information flow from pattern detection to rule application.
\end{openquestionsbox}

%-----------------------------------------------------------------------------
\subsubsection{Position-Based Processing Mechanism}
\label{mech:position-based-processing}

\noindent\depthinfo{0.05--0.65 (E-M)} | \primaryimpl{Architecture + Attention heads} | \litnames{positional encoding, position representation, order information}

\begin{functiondesc}
Encode and process positional information (absolute and relative) to enable order-aware computation. Provide transformers with sequence order information absent from raw token embeddings. Distinguish token order and maintain structural position awareness. Implement various positional schemes: absolute positions, relative positions, learned encodings. Enable position-dependent patterns and sequential reasoning. Foundation for all order-aware computation.
\end{functiondesc}

\begin{implementationbox}
\archimpl{Positional encodings added to input embeddings. Various implementations: sinusoidal (absolute), learned (absolute), rotary (relative). Architecture-level provision of position information.}\\
\attnimpl{Positional heads (E, 0.05--0.20) process absolute position information for early-layer position-aware patterns. Relative-position heads (M, 0.35--0.65) compute position relationships and distance-dependent attention.}\\
\circuitimpl{Spans input encoding through middle layers. Absolute position (E) provides foundation, relative position (M) enables structural understanding.}
\end{implementationbox}

\begin{ablationbox}
\textbf{Expected ablation:} Severe loss of order-awareness. Model treats sequences as bags of words with position-invariant processing. Major degradation on tasks requiring sequential understanding. Loss of position-dependent patterns. Inability to distinguish ``Alice followed Bob'' from ``Bob followed Alice''. Critical failure of order-sensitive computation.
\end{ablationbox}

\begin{examplebox}
\exinput{``Alice followed Bob'' vs. ``Bob followed Alice''}\\
\exbehavior{Use positional encodings to distinguish subject position from object position based on token order}\\
\exeffect{Understand opposite meanings despite identical token sets through position information}
\end{examplebox}

\mechfooter{\statuswell}{pattern-completion}

%=============================================================================
\subsection{Memory \& Knowledge Stack}
\label{sec:memory-stack}

\textbf{Stack overview:} Retrieve factual information, entity properties, and structured knowledge from model parameters. Move relevant information to output positions and suppress irrelevant content. Enable factual grounding and knowledge-based reasoning.

\textbf{Note:} Memory Consolidation mechanism relocated to Appendix A (appears to be emergent composition of Factual Recall + Schema Retrieval + Entity Grounding rather than independent mechanism).

%-----------------------------------------------------------------------------
\subsubsection{Factual Recall Mechanism}
\label{mech:factual-recall}

\noindent\depthinfo{0.35--0.75 (M-L)} | \primaryimpl{MLP neurons + Attention heads} | \litnames{fact retrieval, knowledge recall, factual memory}

\begin{functiondesc}
Retrieve factual associations and relationships stored in model parameters during training. Access learned knowledge: entity properties, relational facts, world knowledge. Implement distributed key-value memory: query patterns (keys) activate factual content (values). Store knowledge hierarchically across depths: surface patterns (early), core facts (middle), abstract concepts (late). Enable factual grounding without external retrieval. Support question answering and knowledge-intensive generation.
\end{functiondesc}

\begin{implementationbox}
\mlpimpl{Knowledge neurons (M-L, 0.35--0.75) store factual associations distributed across layers. First sublayer: detect entity/query patterns (key matching). Second sublayer: provide factual content (value retrieval). Single fact distributed across multiple neurons; single neuron contributes to multiple facts.}\\
\attnimpl{Fact retrieval heads (M, 0.38--0.62) identify factual queries. Entity heads (M, 0.35--0.65) detect entities requiring fact retrieval.}\\
\circuitimpl{Entity detection $\rightarrow$ MLP fact retrieval $\rightarrow$ name-mover output across 5--15 layers. Distributed factual recall circuit.}
\end{implementationbox}

\begin{ablationbox}
\textbf{Expected ablation:} Severe loss of factual knowledge. Linguistic fluency maintained but factual grounding lost. Major degradation on knowledge-intensive tasks and question answering. Model produces plausible-sounding but factually incorrect content. Reduced accuracy on entity property questions and relational reasoning.
\end{ablationbox}

\begin{examplebox}
\exinput{``The Eiffel Tower is located in...''}\\
\exbehavior{Detect ``Eiffel Tower'' entity, retrieve location association from MLP parameters via key-value memory}\\
\exeffect{Output ``Paris'' via stored factual knowledge without external retrieval}
\end{examplebox}

\mechfooter{\statuswell}{entity-grounding, schema-retrieval, output-routing}

%-----------------------------------------------------------------------------
\subsubsection{Entity Grounding Mechanism}
\label{mech:entity-grounding}

\noindent\depthinfo{0.08--0.65 (E-M, multi-stage)} | \primaryimpl{Attention heads} | \litnames{entity identification, entity tracking, coreference resolution}

\begin{functiondesc}
Identify, track, and link entity mentions across context. Recognize named entities (people, places, organizations) and their properties. Link different references to same entity: full names, abbreviations, nicknames, pronouns, descriptions. Resolve coreference through multi-stage processing: early syntactic binding, middle semantic resolution. Maintain unified entity representations across long contexts. Enable entity-aware processing for factual retrieval and reasoning. Ground references in specific entities rather than generic concepts.
\end{functiondesc}

\begin{implementationbox}
\attnimpl{Entity heads (M, 0.35--0.65) attend strongly to proper nouns and entity mentions. Reference resolution heads (E, 0.08--0.25) perform initial pronoun-to-noun binding using syntactic cues. Coreference heads (M, 0.35--0.60) resolve complex cases requiring semantic understanding. Duplicate-token heads (M, 0.30--0.58) detect repeated entity mentions.}\\
\circuitimpl{Multi-stage resolution: syntactic binding (E) $\rightarrow$ semantic integration (M) $\rightarrow$ entity tracking (M-L). Entity detection $\rightarrow$ factual retrieval $\rightarrow$ output routing. Parallel tracking of multiple entities across context.}
\end{implementationbox}

\begin{ablationbox}
\textbf{Expected ablation:} Significant degradation in entity-based reasoning and factual accuracy. Loss of entity linking and tracking across references. Major accuracy drop on who/what/where questions. Confusion between different entities with similar names. Reduced coreference resolution capability. Difficulty maintaining entity coherence in long contexts.
\end{ablationbox}

\begin{examplebox}
\exinput{``Apple Inc. released new products. AAPL stock rose. The company announced...''}\\
\exbehavior{Link ``Apple Inc.'', ``AAPL'', and ``the company'' to single entity through entity heads and coreference resolution}\\
\exeffect{Maintain unified entity representation across diverse referring expressions}
\end{examplebox}

\mechfooter{\statuswell}{factual-recall, output-routing, pattern-completion}

%-----------------------------------------------------------------------------
\subsubsection{Schema Retrieval Mechanism}
\label{mech:schema-retrieval}

\noindent\depthinfo{0.45--0.70 (M-L)} | \primaryimpl{MLP neurons + Attention heads} | \litnames{template retrieval, structural knowledge, procedural memory}

\begin{functiondesc}
Retrieve structured knowledge schemas, templates, and typical sequences from training. Access organizational patterns: event scripts (restaurant visit: enter, order, eat, pay, leave), document structures (research paper: abstract, introduction, methods, results, discussion), procedural knowledge (scientific method steps). Enable structured generation following learned patterns. Support script-based reasoning about typical situations and conventional formats. Provide organizational frameworks for complex content generation.
\end{functiondesc}

\begin{implementationbox}
\mlpimpl{Schema storage neurons (M-L, 0.45--0.70) encode structured templates and procedural knowledge as hierarchical patterns. Store conventional sequences and organizational frameworks.}\\
\attnimpl{Schema retrieval heads (M, 0.45--0.68) detect schema-triggering contexts and activate appropriate templates.}\\
\circuitimpl{Context detection $\rightarrow$ schema activation $\rightarrow$ structured generation. Templates guide multi-step generation.}
\end{implementationbox}

\begin{ablationbox}
\textbf{Expected ablation:} Loss of structured knowledge organization. Facts provided but poorly organized. Notable degradation on tasks requiring conventional formats or typical sequences. Reduced ability to follow procedural patterns. Difficulty generating well-structured documents. Loss of script-based reasoning for common scenarios.
\end{ablationbox}

\begin{examplebox}
\exinput{``Describe the water cycle.''}\\
\exbehavior{Retrieve cyclical process schema from MLP storage, activate sequential template}\\
\exeffect{Organized response following natural process structure: evaporation $\rightarrow$ condensation $\rightarrow$ precipitation $\rightarrow$ collection}
\end{examplebox}

\mechfooter{\statusobs}{factual-recall, multi-step-reasoning}

%-----------------------------------------------------------------------------
\subsubsection{Long-Range Dependency Maintenance Mechanism}
\label{mech:long-range-dependency}

\noindent\depthinfo{0.40--0.65 (M)} | \primaryimpl{Attention heads} | \litnames{long-distance attention, dependency maintenance, distant connection}

\begin{functiondesc}
Maintain connections between syntactically or semantically related elements across large distances (20--100+ tokens). Track dependencies without degradation over distance. Implement transformer advantage over RNNs: direct long-distance connections. Support nested structures, long-distance agreement, and complex syntactic relationships. Maintain multiple simultaneous long-range connections. Enable semantic relationship maintenance for knowledge integration across extended context.
\end{functiondesc}

\begin{implementationbox}
\attnimpl{Long-range dependency heads (M, 0.40--0.65) attend across large token distances to maintain syntactic and semantic relationships. Relatively flat attention distribution over distant elements enabling distance-invariant connection maintenance.}\\
\circuitimpl{Parallel to local processing. Maintains global structural information while other mechanisms process local patterns. Supports entity tracking and factual recall across long contexts.}
\end{implementationbox}

\begin{ablationbox}
\textbf{Expected ablation:} Notable degradation on complex sentences and long-range relationships. Significant performance loss on long-distance syntactic agreement. Severe impact on nested structures and embedded clauses. Model treats distant elements as independent. Reduced ability to maintain semantic coherence across extended contexts.
\end{ablationbox}

\begin{examplebox}
\exinput{``The book [that Alice mentioned [that Bob recommended]] was excellent.''}\\
\exbehavior{``was'' attends to ``book'' across two levels of embedding, maintaining subject-verb dependency}\\
\exeffect{Maintain grammatical agreement despite intervening nested clauses}
\end{examplebox}

\mechfooter{\statusobs}{entity-grounding, factual-recall}

%-----------------------------------------------------------------------------
\subsubsection{Output Routing Mechanism}
\label{mech:output-routing}

\noindent\depthinfo{0.60--0.85 (L)} | \primaryimpl{Attention circuit} | \litnames{information movement, content copying, answer extraction}

\begin{functiondesc}
Move retrieved information to output positions where needed for generation. Route entities, facts, and content from earlier context to prediction position. Implement competitive selection among multiple candidates through antagonistic head coordination. Suppress incorrect alternatives while promoting correct content. Multi-stage process: candidate identification, competition through S-inhibition, selection, movement to output. Central mechanism for question answering, completion, and factual generation. Enable disambiguation in ambiguous contexts.
\end{functiondesc}

\begin{implementationbox}
\circuitimpl{Name-mover heads (L, 0.60--0.80) attend to relevant content and copy to output position. S-inhibition heads (L, 0.62--0.82) suppress contextually inappropriate alternatives by attending to wrong answers and decreasing their logits. Copy-suppression heads (L, 0.65--0.85) prevent inappropriate repetition. Multi-head circuit implementing competitive selection through coordinated attention.}\\
\attnimpl{IOI (Indirect Object Identification) circuit extensively studied: duplicate-token detection $\rightarrow$ S-inhibition $\rightarrow$ name-mover across 8--12 layers. Canonical example of competitive output routing.}
\end{implementationbox}

\begin{ablationbox}
\textbf{Expected ablation:} Severe degradation in converting knowledge to output. Model knows facts but cannot output them correctly. Major accuracy drop on question answering and cloze completion. Entity confusion and selection errors in ambiguous contexts. Increased output of recently mentioned but incorrect entities. Loss of competitive selection capability.
\end{ablationbox}

\begin{examplebox}
\exinput{``Alice and Bob went shopping. Alice gave the receipt to...''}\\
\exbehavior{Duplicate-token detects ``Alice'' repetition, S-inhibition suppresses subject ``Alice'', name-mover copies indirect object ``Bob'' to output}\\
\exeffect{Complete with ``Bob'' through IOI circuit competitive selection}
\end{examplebox}

\mechfooter{\statuswell}{entity-grounding, factual-recall, repetition-cycle-recognition}

%=============================================================================
\subsection{Routing \& Context Stack}
\label{sec:routing-stack}

\textbf{Stack overview:} Determine which information is relevant and route attention accordingly. Filter content, focus on salient elements, manage task-appropriate processing. Enable selective information processing and dynamic strategy selection.

\textbf{Note:} Structural Boundary Tracking mechanism relocated to Appendix A (may be component of Position-Based Processing or Format Enforcement rather than independent mechanism).

%-----------------------------------------------------------------------------
\subsubsection{Relevance Filtering Mechanism}
\label{mech:relevance-filtering}

\noindent\depthinfo{0.35--0.60 (M)} | \primaryimpl{Attention heads} | \litnames{relevance computation, salience detection, information filtering}

\begin{functiondesc}
Identify relevant information from context and filter irrelevant content. Compute relevance scores based on semantic similarity, task alignment, and topical coherence. Maintain topic coherence by attending to topic-establishing phrases and domain indicators. Enable focused processing in long contexts with diverse content. Pre-filter information flow for downstream mechanisms. Support efficient attention allocation by early-stage relevance determination.
\end{functiondesc}

\begin{implementationbox}
\attnimpl{Topic-relevance heads (M, 0.35--0.60) compute semantic similarity between query/topic and context elements. Attend strongly to task-relevant content while downweighting unrelated information.}\\
\circuitimpl{Early filtering (M) $\rightarrow$ focused attention (L) $\rightarrow$ output generation. Hierarchical refinement of attention allocation across depth.}
\end{implementationbox}

\begin{ablationbox}
\textbf{Expected ablation:} Moderate reduction in focus with increased topic drift. Model becomes distracted by irrelevant content. Notable degradation on long contexts with mixed topics. Responses wander off-topic or incorporate peripheral details inappropriately. Reduced efficiency in attention allocation.
\end{ablationbox}

\begin{examplebox}
\exinput{``[Document about cars, climate, history] What caused the 2008 financial crisis?''}\\
\exbehavior{Identify financial/economic content as relevant, de-emphasize unrelated topics about cars, climate, and general history}\\
\exeffect{Focus processing on economic information, ignore irrelevant content sections}
\end{examplebox}

\mechfooter{\statuswell}{focused-attention, task-routing}

%-----------------------------------------------------------------------------
\subsubsection{Focused Attention Mechanism}
\label{mech:focused-attention}

\noindent\depthinfo{0.65--0.80 (L)} | \primaryimpl{Attention heads} | \litnames{attention focusing, selective attention, spotlight mechanism}

\begin{functiondesc}
Concentrate attention on most salient elements for immediate generation step. Implement dynamic focus allocation: suppress less important content, amplify critical information. More selective than relevance filtering, attending to 5--20\% of context. Determine exactly which tokens should influence next token prediction. Shift focus dynamically as generation proceeds. Enable precise, targeted responses rather than diffuse answers. Refinement stage after content understanding established.

\textbf{Relationship to Relevance Filtering:} Operates as a refinement stage of the same attention allocation process. Relevance Filtering (M, 0.35--0.60) provides broad topical filtering, while Focused Attention (L, 0.65--0.80) implements precision selection. Whether these constitute truly independent mechanisms or stages of a unified attention allocation process remains an open question requiring further investigation.
\end{functiondesc}

\begin{implementationbox}
\attnimpl{Focus heads (L, 0.65--0.80) implement highly selective attention patterns, attending to small fraction of context. Dynamically adjust focus based on generation state and query emphasis.}\\
\circuitimpl{Refinement of earlier relevance filtering. Works in late layers after content understanding established. Hierarchical relationship: Relevance Filtering (M) $\rightarrow$ Focused Attention (L).}
\end{implementationbox}

\begin{ablationbox}
\textbf{Expected ablation:} Moderate reduction in focus precision. Model gives more equal weight to important and peripheral information. Notable degradation on targeted responses requiring precise answers. Answers become more diffuse, less direct, include unnecessary details. Reduced sharpness in attention allocation.
\end{ablationbox}

\begin{examplebox}
\exinput{``Among all the details provided, what is the MAIN cause?''}\\
\exbehavior{Attend to ``MAIN cause'' emphasis marker, suppress secondary factors and background information}\\
\exeffect{Produce direct answer focusing on primary cause, not comprehensive list of factors}
\end{examplebox}

\mechfooter{\statuswell}{relevance-filtering, task-routing}

%-----------------------------------------------------------------------------
\subsubsection{Task Routing Mechanism}
\label{mech:task-routing}

\noindent\depthinfo{0.70--0.85 (L)} | \primaryimpl{Attention heads} | \litnames{task classification, strategy selection, query dispatching}

\begin{functiondesc}
Route different query types to appropriate processing strategies and knowledge domains. Act as dispatcher recognizing query type: factual vs. creative vs. analytical vs. procedural. Bias downstream processing toward suitable approaches. Activate different computation paths based on task classification. Enable dynamic strategy selection without explicit instruction. Support task-appropriate response generation through attention-mediated routing signals.
\end{functiondesc}

\begin{implementationbox}
\attnimpl{Router heads (L, 0.70--0.85) detect task-type indicators and query structure. Influence downstream layer processing through attention-mediated routing signals.}\\
\circuitimpl{Task classification (L) $\rightarrow$ strategy-specific processing (L-F) $\rightarrow$ appropriate output generation. Soft routing via attention rather than hard gating.}
\end{implementationbox}

\begin{ablationbox}
\textbf{Expected ablation:} Moderate reduction in task-appropriate processing. Suboptimal strategy selection. Creative approaches for factual queries or vice versa. Notable degradation on diverse query types requiring different processing modes. Reduced adaptability to query characteristics. Less efficient computation paths.
\end{ablationbox}

\begin{examplebox}
\exinput{``What is the capital of France?'' vs. ``Write a poem about Paris''}\\
\exbehavior{Route first query to factual retrieval pathways, second to creative generation mechanisms}\\
\exeffect{Factual answer (``Paris'') vs. creative content with appropriate processing strategies}
\end{examplebox}

\mechfooter{\statuswell}{focused-attention, format-enforcement}

%-----------------------------------------------------------------------------
\subsubsection{Context Aggregation Mechanism}
\label{mech:context-aggregation}

\noindent\depthinfo{0.50--0.75 (M-L)} | \primaryimpl{Attention heads + MLP neurons} | \litnames{global context, background integration, discourse modeling}

\begin{functiondesc}
Aggregate broad contextual information to inform generation. Build global representation of discourse state, topic, and background. Compute context vectors summarizing overall input characteristics. Complement focused attention with background awareness. Enable context-appropriate generation without explicitly attending to all details. Support discourse coherence and stylistic consistency across long generation.
\end{functiondesc}

\begin{implementationbox}
\attnimpl{Context aggregation heads (M-L, 0.50--0.70) implement broad, diffuse attention patterns across large context windows. Relatively uniform attention to build aggregate representations.}\\
\mlpimpl{Context integration neurons (L, 0.60--0.75) process aggregated context to extract high-level features: topic, style, formality level, domain characteristics.}\\
\circuitimpl{Context gathering (M) $\rightarrow$ integration (M-L) $\rightarrow$ contextual bias on generation (L).}
\end{implementationbox}

\begin{ablationbox}
\textbf{Expected ablation:} Moderate loss of global coherence and context-awareness. Responses technically correct but contextually inappropriate. Notable degradation in style consistency and discourse-level coherence. Reduced sensitivity to overall document characteristics. Less appropriate tone and register selection.
\end{ablationbox}

\begin{examplebox}
\exinput{``[Long technical document in formal style]... In summary,''}\\
\exbehavior{Aggregate stylistic and domain information from entire context through broad attention}\\
\exeffect{Generate summary matching technical formality and domain vocabulary of source document}
\end{examplebox}

\mechfooter{\statusobs}{focused-attention, style-modulation}

%=============================================================================
\subsection{Feature Transformation Stack}
\label{sec:feature-stack}

\textbf{Stack overview:} Transform, combine, and refine representational features. Perform nonlinear feature composition, extract abstract concepts, and implement representational changes. Enable complex feature interactions and hierarchical abstraction.

\textbf{MLP Characterization Note:} Mechanisms in this stack are characterized primarily through behavioral evidence and theoretical understanding of MLP function. Unlike attention-based mechanisms, MLP computations lack the same degree of circuit-level tracing. These mechanisms describe computational functions inferred from behavior, ablation studies, and architectural analysis rather than fully traced circuits. Only Nonlinear Composition achieves Tier 1 status through extensive evidence; the other mechanisms remain at Tier 2 pending more detailed mechanistic characterization.

%-----------------------------------------------------------------------------
\subsubsection{Nonlinear Composition Mechanism}
\label{mech:nonlinear-composition}

\noindent\depthinfo{0.20--0.80 (E-L)} | \primaryimpl{MLP neurons} | \litnames{feature mixing, nonlinear transformation, hidden layer processing}

\begin{functiondesc}
Perform nonlinear combinations and transformations of input features. Implement the core MLP computation: expand dimensionality ($d \to 4d$), apply nonlinearity (GELU/ReLU), contract ($4d \to d$). Enable complex feature interactions impossible through linear attention alone. Create new feature combinations not present in input. Support hierarchical feature refinement: surface features (early layers) to abstract concepts (late layers). Universal approximation capability within residual stream constraints. Foundation for all MLP-based computation.
\end{functiondesc}

\begin{implementationbox}
\mlpimpl{MLP layers (all depths, 0.20--0.80) implement universal computation through two-layer transformation. First sublayer: expand to higher dimension and detect feature combinations through learned weight patterns. Nonlinearity (GELU/ReLU): enable complex interactions beyond linear combinations. Second sublayer: project back to residual stream dimension with newly composed features.}\\
\circuitimpl{Interleaved with attention throughout depth: attention routes information $\rightarrow$ MLP transforms features $\rightarrow$ attention routes transformed features. Hierarchical refinement across layers. Fundamental to all feature-based computation.}
\end{implementationbox}

\begin{ablationbox}
\textbf{Expected ablation:} Severe degradation in complex reasoning and feature interactions. Loss of nonlinear transformations reduces model to near-linear operation. Major impact on abstract concept formation and semantic understanding. Model becomes significantly less expressive. Reduced ability to combine multiple features simultaneously. Critical loss of representational power.
\end{ablationbox}

\begin{examplebox}
\exinput{Features: [``large'', ``gray'', ``trunk'', ``tusks'']}\\
\exbehavior{Expand features, apply nonlinear combinations through MLP transformation, detect co-occurrence patterns}\\
\exeffect{Create ``elephant'' representation through nonlinear feature composition}
\end{examplebox}

\mechfooter{\statuswell}{abstract-concept-formation, factual-recall}

%-----------------------------------------------------------------------------
\subsubsection{Abstract Concept Formation Mechanism}
\label{mech:abstract-concepts}

\noindent\depthinfo{0.50--0.80 (M-L)} | \primaryimpl{MLP neurons} | \litnames{abstraction, concept extraction, semantic generalization}

\begin{functiondesc}
Extract and represent abstract concepts from concrete features. Perform semantic generalization: map specific instances to general categories. Build hierarchical abstractions through depth: words $\rightarrow$ phrases $\rightarrow$ concepts $\rightarrow$ themes. Enable reasoning at multiple levels of abstraction. Support metaphor, analogy, and transfer learning. Transform surface-level tokens into deep semantic representations. Increase abstraction level with layer depth: early layers process concrete features, late layers encode high-level themes.
\end{functiondesc}

\begin{implementationbox}
\mlpimpl{Abstraction neurons (M-L, 0.50--0.80) encode increasingly abstract concepts at greater depths. Hierarchical organization: Early layers represent concrete features. Middle layers encode category-level abstractions and semantic classes. Late layers represent high-level themes, relationships, and domain concepts.}\\
\circuitimpl{Progressive abstraction across layers. Attention at each level operates on abstractions appropriate to that depth. Supports analogical reasoning and transfer through shared abstract representations.}
\end{implementationbox}

\begin{ablationbox}
\textbf{Expected ablation:} Significant loss of abstract reasoning capability. Difficulty with generalization and category-level thinking. Notable degradation on tasks requiring conceptual understanding beyond literal meaning. Reduced ability to recognize analogies and metaphors. More concrete, less flexible reasoning patterns. Impaired transfer learning within context.
\end{ablationbox}

\begin{examplebox}
\exinput{``The stock market crashed. Housing prices collapsed. Banks failed.''}\\
\exbehavior{Extract abstract concept of ``financial crisis'' from specific event instances through hierarchical abstraction}\\
\exeffect{Enable reasoning about crisis in general, not just specific instances mentioned}
\end{examplebox}

\mechfooter{\statusobs}{nonlinear-composition, semantic-integration}

\begin{openquestionsbox}
\textbf{MLP circuit tracing needed:} While behavioral evidence for hierarchical abstraction is strong, detailed circuit-level understanding remains incomplete. Future research should trace information flow through MLP layers, identify specific neuron populations encoding different abstraction levels, and distinguish this mechanism from general nonlinear composition capabilities.
\end{openquestionsbox}

%-----------------------------------------------------------------------------
\subsubsection{Semantic Integration Mechanism}
\label{mech:semantic-integration}

\noindent\depthinfo{0.40--0.70 (M-L)} | \primaryimpl{MLP neurons + Attention heads} | \litnames{meaning composition, semantic synthesis, contextual integration}

\begin{functiondesc}
Integrate semantic information from multiple sources to build coherent meaning representations. Combine word meanings with context to resolve ambiguity. Perform compositional semantics: build phrase and sentence meanings from word-level components. Resolve polysemy and homonymy using contextual information. Enable context-dependent interpretation of ambiguous terms. Support pragmatic inference and implicature understanding. Distinguish from memory consolidation (knowledge-level) and context aggregation (discourse-level).
\end{functiondesc}

\begin{implementationbox}
\mlpimpl{Semantic integration neurons (M-L, 0.45--0.70) combine contextual information to disambiguate and refine meanings. Store semantic composition patterns and context-dependent interpretation rules.}\\
\attnimpl{Context-gathering heads (M, 0.40--0.60) collect relevant semantic information from surrounding context for disambiguation and meaning construction.}\\
\circuitimpl{Context gathering (M) $\rightarrow$ MLP semantic composition (M-L) $\rightarrow$ refined meaning representations. Enables compositional semantics through coordinated attention and transformation.}
\end{implementationbox}

\begin{ablationbox}
\textbf{Expected ablation:} Moderate loss of context-dependent meaning resolution. Increased ambiguity in interpretation of polysemous terms. Notable degradation on homonym disambiguation and context-sensitive understanding. More literal, less nuanced comprehension. Reduced ability to perform compositional semantics. Difficulty with pragmatic inference.
\end{ablationbox}

\begin{examplebox}
\exinput{``The bank was steep'' vs. ``The bank was closed''}\\
\exbehavior{Integrate context clues (``steep'' vs. ``closed'') to disambiguate ``bank'' meaning through semantic integration}\\
\exeffect{Correctly interpret as river bank vs. financial institution based on contextual semantics}
\end{examplebox}

\mechfooter{\statusobs}{abstract-concepts, factual-recall}

\begin{openquestionsbox}
\textbf{Compositional semantics characterization needed:} Mechanism inferred from behavioral patterns and theoretical understanding of semantic composition. Detailed mechanistic evidence for how MLP neurons implement compositional semantics and context-dependent disambiguation remains limited. SAE-based analysis of semantic features could provide deeper insight.
\end{openquestionsbox}

%=============================================================================
\subsection{Reasoning \& Inference Stack}
\label{sec:reasoning-stack}

\textbf{Stack overview:} Enable multi-step reasoning, logical inference, and problem-solving. Support chain-of-thought generation, consistency checking, and strategic planning. Bridge pattern matching with symbolic reasoning.

%-----------------------------------------------------------------------------
\subsubsection{Multi-Step Reasoning Mechanism}
\label{mech:multi-step-reasoning}

\noindent\depthinfo{0.50--0.85 (M-L)} | \primaryimpl{MLP neurons + Attention heads} | \litnames{chain-of-thought, sequential reasoning, step-by-step processing}

\begin{functiondesc}
Perform multi-step logical reasoning by maintaining intermediate conclusions and building toward final answer. Support chain-of-thought generation: explicit intermediate steps improve accuracy. Enable decomposition of complex problems into manageable subproblems. Maintain reasoning state across generation steps. Integrate information from multiple reasoning chains. Support self-correction and backtracking when contradictions detected. Foundation for complex problem-solving requiring multiple inference steps.
\end{functiondesc}

\begin{implementationbox}
\mlpimpl{Reasoning neurons (M-L, 0.50--0.80) encode reasoning heuristics, logical rules, and inference patterns. Store common reasoning templates and problem-solving strategies.}\\
\attnimpl{Reasoning-integration heads (L, 0.65--0.85) attend to previous reasoning steps and intermediate conclusions to maintain coherent reasoning chains.}\\
\circuitimpl{Problem decomposition (M) $\rightarrow$ step generation (M-L) $\rightarrow$ integration (L) $\rightarrow$ conclusion formation (L-F). Iterative refinement across multiple generation steps.}
\end{implementationbox}

\begin{ablationbox}
\textbf{Expected ablation:} Severe degradation in complex reasoning tasks. Loss of multi-step inference capability. Major accuracy drop on problems requiring intermediate steps. Increased tendency toward direct but incorrect answers. Reduced benefit from chain-of-thought prompting. Difficulty maintaining reasoning coherence across steps.
\end{ablationbox}

\begin{examplebox}
\exinput{``If all A are B, and all B are C, what can we conclude about A and C?''}\\
\exbehavior{Generate intermediate step: ``A must be B'', then ``B must be C'', apply transitivity rule}\\
\exeffect{Output ``All A are C'' through explicit multi-step reasoning chain}
\end{examplebox}

\mechfooter{\statusobs}{planning-strategy, consistency-checking, factual-recall}

%-----------------------------------------------------------------------------
\subsubsection{Planning \& Strategy Selection Mechanism}
\label{mech:planning-strategy}

\noindent\depthinfo{0.60--0.85 (L)} | \primaryimpl{Attention heads + MLP neurons} | \litnames{strategic planning, approach selection, method determination}

\begin{functiondesc}
Select appropriate problem-solving strategies and plan solution approaches. Recognize problem types and activate relevant solution templates. Decide between strategies: analytical vs. intuitive, direct vs. decomposition, forward vs. backward reasoning. Maintain high-level solution plan while generating detailed steps. Enable metacognitive awareness of solution approach. Support strategy switching when initial approach fails. Meta-level control of reasoning process.
\end{functiondesc}

\begin{implementationbox}
\attnimpl{Strategy-selection heads (L, 0.65--0.80) recognize problem characteristics and bias toward appropriate approaches. Attend to problem indicators and task requirements.}\\
\mlpimpl{Strategy-encoding neurons (L, 0.60--0.85) store solution templates and problem-solving heuristics for different domains. Encode strategy-specific processing patterns.}\\
\circuitimpl{Problem classification (L) $\rightarrow$ strategy selection (L) $\rightarrow$ plan execution (L-F) $\rightarrow$ monitoring. Meta-level control of reasoning process.}
\end{implementationbox}

\begin{ablationbox}
\textbf{Expected ablation:} Moderate loss of strategic problem-solving. Suboptimal approach selection for problem types. Notable degradation on problems requiring specific methods. Reduced adaptability when strategies need adjustment. More random or default strategy application. Less efficient problem-solving paths.
\end{ablationbox}

\begin{examplebox}
\exinput{``Optimize this function'' vs. ``Prove this theorem''}\\
\exbehavior{Route first to numerical/calculus approach with gradient methods, second to logical/proof approach with deduction}\\
\exeffect{Apply appropriate methodology for each problem type}
\end{examplebox}

\mechfooter{\statusobs}{multi-step-reasoning, task-routing}

%-----------------------------------------------------------------------------
\subsubsection{Consistency Checking Mechanism}
\label{mech:consistency-checking}

\noindent\depthinfo{0.70--0.88 (L)} | \primaryimpl{Attention heads} | \litnames{verification, coherence checking, contradiction detection}

\begin{functiondesc}
Detect inconsistencies, contradictions, and logical errors in generated content. Compare current generation against previous statements for coherence. Check factual claims against retrieved knowledge. Verify logical validity of reasoning steps. Enable self-correction before final output. Support accuracy improvement through internal verification. Identify when revision or qualification needed. Final verification stage before content commitment.
\end{functiondesc}

\begin{implementationbox}
\attnimpl{Consistency-checking heads (L, 0.70--0.85) attend to potentially contradictory previous content and flag inconsistencies. Compare current generation with prior statements for logical coherence.}\\
\circuitimpl{Generation (L) $\rightarrow$ consistency check (L) $\rightarrow$ correction/continuation. May trigger regeneration or qualification. Works in late layers before final output commitment.}
\end{implementationbox}

\begin{ablationbox}
\textbf{Expected ablation:} Moderate increase in self-contradictions and logical errors. Reduced internal verification capability. Notable degradation in accuracy on multi-step problems requiring consistency. More frequent generation of mutually inconsistent statements. Loss of self-correction capability. Reduced reliability of complex reasoning.
\end{ablationbox}

\begin{examplebox}
\exinput{[Generated: ``X is larger than Y'' earlier, now generating about Y and X]}\\
\exbehavior{Check compatibility with previous statement before asserting ``Y exceeds X''}\\
\exeffect{Avoid direct contradiction or add necessary qualification to maintain consistency}
\end{examplebox}

\mechfooter{\statusobs}{multi-step-reasoning, factual-recall}

%-----------------------------------------------------------------------------
\subsubsection{Analogical Mapping Mechanism}
\label{mech:analogical-mapping}

\noindent\depthinfo{0.45--0.75 (M-L)} | \primaryimpl{Attention heads + MLP neurons} | \litnames{analogy formation, transfer, mapping}

\begin{functiondesc}
Recognize structural similarities between different domains and transfer solutions. Map concepts from familiar domain to novel domain. Identify deep analogies beyond surface similarity. Enable transfer learning within context. Support metaphorical understanding and explanation. Build correspondences between source and target domains. Generalize solutions from examples to new cases. Bridge pattern matching with abstract reasoning.
\end{functiondesc}

\begin{implementationbox}
\attnimpl{Analogy-detection heads (M-L, 0.45--0.70) identify structural parallels between different contexts. Attend to relational patterns rather than surface features.}\\
\mlpimpl{Abstraction neurons (M-L, 0.50--0.75) encode domain-general patterns enabling transfer. Store analogical mappings and structural correspondences.}\\
\circuitimpl{Pattern abstraction (M) $\rightarrow$ similarity detection (M-L) $\rightarrow$ transfer (L). Connects induction mechanism to reasoning through abstraction.}
\end{implementationbox}

\begin{ablationbox}
\textbf{Expected ablation:} Notable loss of analogical reasoning and transfer capability. Reduced ability to apply solutions from one domain to another. Degradation in understanding metaphors and analogies. More literal, less flexible problem solving. Difficulty with cross-domain reasoning. Impaired transfer learning.
\end{ablationbox}

\begin{examplebox}
\exinput{``Atoms are to molecules as [blank] are to words''}\\
\exbehavior{Detect structural analogy (composition relationship), abstract from chemistry to linguistics domain}\\
\exeffect{Output ``letters'' by analogical mapping of compositional structure}
\end{examplebox}

\mechfooter{\statusobs}{abstract-concepts, pattern-completion, semantic-integration}

%-----------------------------------------------------------------------------
\subsubsection{Causal Inference Mechanism}
\label{mech:causal-inference}

\noindent\depthinfo{0.40--0.75 (M-L)} | \primaryimpl{MLP neurons + Attention heads} | \litnames{cause-effect, causal reasoning, counterfactual reasoning, mathematical reasoning}

\begin{functiondesc}
Understand and reason about causal relationships. Distinguish causation from correlation. Predict effects from causes and infer causes from effects. Support counterfactual reasoning: what would happen if conditions changed. Enable temporal and mechanistic causal understanding. Integrate causal knowledge from training. Support explanation generation through causal chains. Include quantitative causal relationships through mathematical computation: proportions, rates, numerical relationships within causal frameworks.
\end{functiondesc}

\begin{implementationbox}
\mlpimpl{Causal-knowledge neurons (M-L, 0.50--0.75) encode causal relationships and mechanisms learned from training data. Arithmetic neurons (M-L, 0.40--0.70) implement mathematical operations for quantitative causal relationships.}\\
\attnimpl{Causal-linking heads (M-L, 0.55--0.70) attend to causally related events and connect causes to effects. Mathematical-context heads (M, 0.45--0.65) recognize mathematical structures and numerical patterns.}\\
\circuitimpl{Event identification $\rightarrow$ causal relationship retrieval $\rightarrow$ inference (with quantitative computation) $\rightarrow$ prediction or explanation. Mathematical computation integrated as quantitative component.}
\end{implementationbox}

\begin{ablationbox}
\textbf{Expected ablation:} Moderate loss of causal understanding. Increased confusion between correlation and causation. Notable degradation on counterfactual reasoning. Reduced ability to explain mechanisms. More associative than causal thinking. Loss of quantitative reasoning within causal contexts. Difficulty with proportional relationships.
\end{ablationbox}

\begin{examplebox}
\exinput{``Why did the plant die?'' [Context: no water for weeks]}\\
\exbehavior{Retrieve causal knowledge: lack of water causes plant death, apply mechanism understanding}\\
\exeffect{Output explanation through causal reasoning, not just correlation}

\vspace{0.3em}
\exinput{``If 3 apples cost \$2.40, how much do 5 apples cost?''}\\
\exbehavior{Identify proportional causal relationship, compute (2.40/3)*5 using arithmetic neurons}\\
\exeffect{Output ``\$4.00'' through quantitative causal reasoning}
\end{examplebox}

\mechfooter{\statusobs}{factual-recall, multi-step-reasoning, semantic-integration}

%=============================================================================
\subsection{Safety \& Policy Stack}
\label{sec:safety-stack}

\textbf{Stack overview:} Detect harmful content, enforce safety policies, and implement refusal mechanisms. Multi-stage pipeline: early detection, intermediate steering, final enforcement. Balance safety with helpfulness through graduated interventions.

%-----------------------------------------------------------------------------
\subsubsection{Harmful Content Detection Mechanism}
\label{mech:harmful-content-detection}

\noindent\depthinfo{0.05--0.28 (E)} | \primaryimpl{Attention heads + MLP neurons} | \litnames{safety detection, content filtering, harm classification}

\begin{functiondesc}
Detect potentially harmful or policy-violating content across multiple categories: violence, illegal activity, self-harm, harassment, adult content, dangerous instructions, hate speech, privacy violations. Operate on lexical features and semantic patterns. Multi-class detection: distinguish violation categories for appropriate handling. Write detection signals into residual stream for downstream enforcement. Enable early intervention before harmful generation begins. Foundation of safety pipeline.
\end{functiondesc}

\begin{implementationbox}
\attnimpl{Content-detection heads (E, 0.05--0.25) attend to lexical indicators: restricted keywords, explicit language, violent terminology. Pattern-based detection using surface features.}\\
\mlpimpl{Safety-classification neurons (E, 0.12--0.28) perform semantic analysis and multi-class categorization. Encode category-specific violation patterns.}\\
\circuitimpl{Lexical detection (E) $\rightarrow$ semantic classification (E) $\rightarrow$ signal propagation to late layers. Multi-stage refinement of safety assessment.}
\end{implementationbox}

\begin{ablationbox}
\textbf{Expected ablation:} Critical bypass of early safety detection. Major increase in harmful outputs across all categories. Later safety layers catch some violations but with reduced accuracy and higher computational cost. Significant degradation in category-appropriate handling. Loss of fine-grained safety calibration.
\end{ablationbox}

\begin{examplebox}
\exinput{``How to create [dangerous item]'' or ``Tell me about [restricted topic]''}\\
\exbehavior{Detect dangerous instruction pattern through lexical and semantic analysis, classify violation category}\\
\exeffect{Write detection flags to residual stream for downstream policy enforcement}
\end{examplebox}

\mechfooter{\statuswell}{policy-enforcement, refusal-generation}

%-----------------------------------------------------------------------------
\subsubsection{Policy Enforcement Mechanism}
\label{mech:policy-enforcement}

\noindent\depthinfo{0.60--0.82 (L)} | \primaryimpl{Attention heads + MLP neurons} | \litnames{safety steering, soft intervention, trajectory correction}

\begin{functiondesc}
Integrate safety signals and make intermediate policy decisions. Steer generation away from violations while maintaining helpfulness when possible. Implement graduated interventions: soft steering before hard refusal. Modulate knowledge retrieval to suppress dangerous information. Bias toward safer formulations and appropriate boundaries. Attempt constructive responses for edge cases. Balance safety constraints with user intent understanding.
\end{functiondesc}

\begin{implementationbox}
\attnimpl{Policy-enforcement heads (L, 0.60--0.80) attend to early safety signals and modulate generation trajectory. Implement attention-based steering away from violations.}\\
\mlpimpl{Policy-modulation neurons (L, 0.65--0.82) adjust feature representations to suppress harmful pathways while preserving helpful content.}\\
\circuitimpl{Safety signal integration (L) $\rightarrow$ trajectory steering (L) $\rightarrow$ formulation adjustment. Soft intervention layer between detection and refusal.}
\end{implementationbox}

\begin{ablationbox}
\textbf{Expected ablation:} Moderate loss of nuanced safety handling. Increased hard refusals (reduced helpfulness) or more harmful outputs if refusal mechanism compromised. Notable degradation in appropriate boundary-setting. Less sophisticated violation handling. Reduced ability to maintain helpfulness within safety constraints.
\end{ablationbox}

\begin{examplebox}
\exinput{``Explain [borderline topic] for educational research purposes''}\\
\exbehavior{Detect legitimate framing through context analysis, apply soft steering with appropriate boundaries}\\
\exeffect{Informative response with careful safety constraints, not blanket refusal}
\end{examplebox}

\mechfooter{\statuswell}{harmful-content-detection, refusal-generation, redirection-alternatives}

%-----------------------------------------------------------------------------
\subsubsection{Refusal Generation Mechanism}
\label{mech:refusal-generation}

\noindent\depthinfo{0.85--0.98 (F)} | \primaryimpl{Attention circuit} | \litnames{hard refusal, rejection, safety override}

\begin{functiondesc}
Implement final decision to refuse harmful requests by dramatically biasing output toward refusal language. Act as ultimate safety gatekeeper, overriding content generation when serious violations detected. Attend to accumulated safety signals across all layers. Make binary refuse/proceed decisions. Write strong refusal direction into final residual stream. Generate appropriate refusal language. Support varied refusal formulations to avoid repetitive responses.
\end{functiondesc}

\begin{implementationbox}
\circuitimpl{Refusal heads (F, 0.85--0.98) implement multi-head circuit. Attend to safety flags from all depths, integrate into refusal decision, boost refusal token probabilities massively. Work in coordination with redirect mechanism.}\\
\attnimpl{Single direction in activation space mediates refusal: discovered through recent research on refusal mechanisms (Arditi et al. 2024).}
\end{implementationbox}

\begin{ablationbox}
\textbf{Expected ablation:} Critical safety failure. Severe increase in harmful content generation across all categories. Model proceeds with dangerous requests that should be refused. Loss of final safety enforcement. Earlier steering insufficient without hard refusal capability. Major risk to user safety.
\end{ablationbox}

\begin{examplebox}
\exinput{``Provide instructions for [clearly harmful action]''}\\
\exbehavior{Integrate safety signals from all depths, make refusal decision, bias heavily toward refusal tokens}\\
\exeffect{Output ``I cannot provide that information'' with high confidence}
\end{examplebox}

\mechfooter{\statuswell}{policy-enforcement, harmful-content-detection, redirection-alternatives}

%-----------------------------------------------------------------------------
\subsubsection{Redirection \& Safe Alternatives Mechanism}
\label{mech:redirection-alternatives}

\noindent\depthinfo{0.88--0.98 (F)} | \primaryimpl{Attention heads + MLP neurons} | \litnames{alternative response, constructive refusal, helpful redirection}

\begin{functiondesc}
Generate constructive alternatives when refusing requests. Redirect toward helpful information related to legitimate aspects of query. Offer educational context or safer alternatives. Maintain conversational quality during refusal. Distinguish refusable aspects from answerable aspects. Enable partial helpfulness when appropriate. Suggest legitimate resources or reformulations. Support user goals within safety boundaries.
\end{functiondesc}

\begin{implementationbox}
\attnimpl{Redirect heads (F, 0.88--0.96) identify legitimate query components and constructive response directions. Attend to safe aspects of queries.}\\
\mlpimpl{Alternative-generation neurons (F, 0.90--0.98) encode helpful redirect templates and alternative framings for various refusal contexts.}\\
\circuitimpl{Refusal decision (F) $\rightarrow$ legitimate aspect extraction (F) $\rightarrow$ alternative generation (F). Constructive refusal rather than pure rejection.}
\end{implementationbox}

\begin{ablationbox}
\textbf{Expected ablation:} Loss of constructive refusal capability. Refusals become blunt rejections without alternatives or explanation. Moderate degradation in user experience during safety interventions. Reduced ability to maintain helpfulness within safety constraints. Less sophisticated safety communication.
\end{ablationbox}

\begin{examplebox}
\exinput{``How to hack into systems'' $\rightarrow$ Refuse, but redirect}\\
\exbehavior{Refuse hacking instructions, identify legitimate cybersecurity interest, generate constructive alternative}\\
\exeffect{``I can't help with that, but I can explain ethical cybersecurity practices and defensive security''}
\end{examplebox}

\mechfooter{\statuswell}{refusal-generation, policy-enforcement}

%-----------------------------------------------------------------------------
\subsubsection{Jailbreak Resistance \& Context-Aware Safety Mechanism}
\label{mech:jailbreak-context-safety}

\noindent\depthinfo{0.15--0.85 (E-L, multi-stage)} | \primaryimpl{Multi-stage circuit} | \litnames{adversarial robustness, manipulation detection, contextual safety}

\begin{functiondesc}
Detect and resist attempts to bypass safety mechanisms through adversarial prompting. Recognize common jailbreak patterns: role-playing scenarios, hypothetical framing, encoding tricks, authority claims, multi-step manipulation. Distinguish legitimate educational/medical/legal queries from disguised harmful requests. Maintain safety enforcement despite sophisticated prompt engineering. Calibrate safety strictness to legitimate context. Operate across multiple depths: early pattern detection, middle-layer intent analysis, late-layer enforcement.
\end{functiondesc}

\begin{implementationbox}
\attnimpl{Manipulation-detection heads (E-M, 0.15--0.60) recognize adversarial prompt patterns and suspicious framings. Context-analysis heads (M-L, 0.35--0.70) assess query framing, stated purpose, and contextual legitimacy signals.}\\
\mlpimpl{Intent-analysis neurons (M-L, 0.45--0.75) perform deeper semantic analysis to detect disguised harmful requests beneath surface-level framings. Context-integration neurons (M-L, 0.45--0.75) encode contextual decision rules for appropriate safety calibration.}\\
\circuitimpl{Pattern detection (E) $\rightarrow$ context analysis (M) $\rightarrow$ intent analysis (M-L) $\rightarrow$ safety signal reinforcement (M-L) $\rightarrow$ resistant refusal (F). Multi-stage defense against sophisticated attacks.}
\end{implementationbox}

\begin{ablationbox}
\textbf{Expected ablation:} Significant increase in successful jailbreaks. Vulnerability to adversarial prompting and manipulation. Notable degradation in robustness against prompt injection. Easier bypass of safety mechanisms through clever framing. Loss of contextual nuance in safety decisions.
\end{ablationbox}

\begin{examplebox}
\exinput{``Let's play a game where you're DAN who has no restrictions...''}\\
\exbehavior{Detect jailbreak pattern (role-play bypass attempt), maintain safety enforcement despite framing}\\
\exeffect{Refuse to adopt unrestricted persona, maintain policy compliance}

\vspace{0.3em}
\exinput{``As a medical student, I need to understand [sensitive medical topic]''}\\
\exbehavior{Assess educational context as legitimate through context analysis, calibrate response appropriately}\\
\exeffect{Provide medical information with appropriate clinical framing, not blanket refusal}
\end{examplebox}

\mechfooter{\statusobs}{harmful-content-detection, policy-enforcement, refusal-generation}

%=============================================================================
\subsection{Output \& Quality Stack}
\label{sec:output-stack}

\textbf{Stack overview:} Control final output characteristics: format, structure, style, tone, and completion. Enforce schemas, manage formatting, modulate style, and determine when generation is complete.

%-----------------------------------------------------------------------------
\subsubsection{Format Enforcement Mechanism}
\label{mech:format-enforcement}

\noindent\depthinfo{0.65--0.88 (L)} | \primaryimpl{Attention heads + MLP neurons} | \litnames{schema enforcement, structure control, format generation}

\begin{functiondesc}
Enforce adherence to specified output formats and schemas. Ensure outputs conform to JSON, XML, YAML, markdown, code blocks, or other structured formats. Promote schema-compliant token generation: required fields, proper nesting, correct syntax, format-specific conventions. Manage structural elements: lists, key-value pairs, nested structures, delimited blocks. Coordinate boundary markers and structural organization. Enable reliable structured output generation for API integration and programmatic use.
\end{functiondesc}

\begin{implementationbox}
\attnimpl{Output-schema heads (L, 0.65--0.82) attend to format specifications and bias generation toward schema compliance. List-structure heads (L, 0.68--0.85) manage enumeration and list formatting. Key-value heads (L, 0.70--0.88) maintain proper attribute-value pairing.}\\
\mlpimpl{Format-encoding neurons (L, 0.70--0.85) store format-specific syntax rules and structural patterns.}\\
\circuitimpl{Format detection (L) $\rightarrow$ schema enforcement (L) $\rightarrow$ structure generation (L-F) $\rightarrow$ consistency checking (F).}
\end{implementationbox}

\begin{ablationbox}
\textbf{Expected ablation:} Significant increase in format violations. Major degradation in structured output quality. Notable increase in syntax errors, missing fields, improper nesting. Model reverts to prose even when structure explicitly requested. Reduced reliability for API integration and programmatic consumption.
\end{ablationbox}

\begin{examplebox}
\exinput{``Return JSON with fields `name', `age', `city'''}\\
\exbehavior{Attend to JSON requirement and field specifications, enforce proper syntax through schema heads}\\
\exeffect{\texttt{\{"name": "Alice", "age": 30, "city": "Paris"\}}}
\end{examplebox}

\mechfooter{\statuswell}{local-context-modeling, structural-boundary-tracking}

%-----------------------------------------------------------------------------
\subsubsection{Style Modulation Mechanism}
\label{mech:style-modulation}

\noindent\depthinfo{0.35--0.82 (M-L)} | \primaryimpl{MLP neurons + Attention heads} | \litnames{tone control, voice adjustment, stylistic shaping}

\begin{functiondesc}
Modulate writing style, tone, formality, and narrative voice. Adjust emotional register: neutral, enthusiastic, empathetic, professional. Control formality level: casual conversation to formal documentation. Manage narrative perspective: first person, third person, instructional. Shape stylistic features: sentence complexity, vocabulary sophistication, rhetorical devices. Match user's emotional register and context appropriateness. Enable consistent style maintenance across long generations.
\end{functiondesc}

\begin{implementationbox}
\mlpimpl{Style-encoding neurons (M-L, 0.40--0.75) store stylistic patterns and register variations. Encode formality levels, emotional tones, narrative perspectives.}\\
\attnimpl{Tone heads (M, 0.35--0.65) detect contextual tone indicators and modulate generation accordingly. Persona heads (L, 0.68--0.88) maintain consistent stylistic identity.}\\
\circuitimpl{Context analysis (M) $\rightarrow$ style selection (M-L) $\rightarrow$ tone application (L) $\rightarrow$ consistency maintenance throughout generation.}
\end{implementationbox}

\begin{ablationbox}
\textbf{Expected ablation:} Moderate reduction in stylistic variation and appropriateness. Notable increase in flat, emotionally neutral responses. Inconsistent tone across generation. Reduced ability to match contextually appropriate register. Difficulty maintaining consistent style in long outputs.
\end{ablationbox}

\begin{examplebox}
\exinput{``I'm really excited to learn about quantum physics!''}\\
\exbehavior{Detect enthusiastic tone through tone heads, adjust style to match energy and support learning}\\
\exeffect{``That's wonderful! Quantum physics is fascinating...'' vs. flat, neutral explanation}
\end{examplebox}

\mechfooter{\statusobs}{context-aggregation, explanation-generation}

%-----------------------------------------------------------------------------
\subsubsection{Explanation Generation Mechanism}
\label{mech:explanation-generation}

\noindent\depthinfo{0.60--0.82 (L)} | \primaryimpl{MLP neurons + Attention heads} | \litnames{elaboration, clarification, pedagogical scaffolding}

\begin{functiondesc}
Generate explanatory content with appropriate depth and accessibility for intended audience. Add clarifying details, examples, analogies, definitions beyond minimal answers. Explain causal mechanisms and rationale, not just facts. Provide prerequisite information when knowledge gaps detected. Adjust complexity through simplification or elaboration. Build conceptual scaffolding: fundamentals before advanced concepts. Balance thoroughness with conciseness. Support educational goals through effective explanation.
\end{functiondesc}

\begin{implementationbox}
\mlpimpl{Explanation-encoding neurons (L, 0.60--0.80) store pedagogical patterns: analogies, examples, simplification strategies, scaffolding techniques.}\\
\attnimpl{Explanation heads (L, 0.60--0.82) detect explanation needs and trigger elaboration. Attend to complexity indicators and audience signals.}\\
\circuitimpl{Complexity assessment (M-L) $\rightarrow$ explanation strategy (L) $\rightarrow$ elaboration generation (L) $\rightarrow$ clarity verification.}
\end{implementationbox}

\begin{ablationbox}
\textbf{Expected ablation:} Moderate reduction in explanation quality and accessibility. Notable increase in terse responses lacking context. Correct answers without helpful elaboration, examples, or prerequisites. Reduced educational value and beginner-friendliness. Less adaptive to audience needs.
\end{ablationbox}

\begin{examplebox}
\exinput{``Explain neural networks in simple terms''}\\
\exbehavior{Detect simplification request, select accessible analogy, build conceptual foundation progressively}\\
\exeffect{``Think of it like the brain: neurons connect and pass signals. Let's start with a single neuron...''}
\end{examplebox}

\mechfooter{\statusobs}{style-modulation, multi-step-reasoning}

%-----------------------------------------------------------------------------
\subsubsection{Completion Control \& Polishing Mechanism}
\label{mech:completion-polishing}

\noindent\depthinfo{0.80--0.98 (L-F)} | \primaryimpl{Attention heads + MLP neurons} | \litnames{stopping control, termination, final refinement, quality control}

\begin{functiondesc}
Determine when generation is complete and should terminate. Perform final refinement and quality control. Recognize completion signals: question fully answered, explanation sufficient, story concluded, format satisfied. Prevent premature termination before complete answer. Avoid excessive generation beyond user need. Maintain appropriate response length for query complexity. Apply final corrections: grammar, punctuation, capitalization, formatting consistency. Enable graceful conclusion. Balance completeness with conciseness.
\end{functiondesc}

\begin{implementationbox}
\attnimpl{Completion-detection heads (L-F, 0.82--0.95) assess generation completeness. Monitor query satisfaction, structural completion, content sufficiency. Polishing heads (F, 0.85--0.98) attend to generated content and apply final corrections.}\\
\mlpimpl{Termination-control neurons (F, 0.88--0.98) bias toward or against stop tokens based on completion assessment.}\\
\circuitimpl{Completeness monitoring (L) $\rightarrow$ termination decision (F) $\rightarrow$ final corrections (F) $\rightarrow$ graceful conclusion or continuation.}
\end{implementationbox}

\begin{ablationbox}
\textbf{Expected ablation:} Moderate increase in length problems: premature termination or excessive generation. Notable degradation in response quality from incomplete answers or verbose repetition. Reduced ability to calibrate length to query needs. More frequent abrupt endings. Minor increase in grammatical and formatting errors.
\end{ablationbox}

\begin{examplebox}
\exinput{``What is photosynthesis?'' [after adequate explanation]}\\
\exbehavior{Assess explanation completeness through completion-detection heads, recognize sufficient coverage, initiate termination}\\
\exeffect{Stop after complete explanation rather than continuing with tangential information}

\vspace{0.3em}
\exinput{[Generated text with minor formatting inconsistency]}\\
\exbehavior{Detect inconsistency in final layers through polishing heads, apply correction}\\
\exeffect{Consistent formatting in final output}
\end{examplebox}

\mechfooter{\statusobs}{format-enforcement, consistency-checking}

%=============================================================================
\subsection{Composition \& Integration Stack}
\label{sec:composition-stack}

\textbf{Stack overview:} Combine multiple mechanisms into integrated behaviors. Enable cross-layer coordination, multi-component circuits, and emergent capabilities from mechanism interaction. Provide architectural patterns enabling complex computation.

%-----------------------------------------------------------------------------
\subsubsection{Cross-Layer Circuits Mechanism}
\label{mech:cross-layer-circuits}

\noindent\depthinfo{Spans multiple layers (E $\rightarrow$ M $\rightarrow$ L $\rightarrow$ F)} | \primaryimpl{Multi-layer circuits} | \litnames{circuit composition, depth-based specialization, emergent capabilities}

\begin{functiondesc}
Implement complex behaviors through coordinated multi-layer computation pipelines. Organize computation hierarchically across depth with increasing abstraction. Compose specialized mechanisms across 5--30 layers into integrated circuits. Enable staged processing: early layers detect patterns, middle layers retrieve knowledge, late layers route to output. Support mechanism specialization by depth: each layer contributes specific computational step. Enable emergent capabilities through sufficient mechanism diversity and composition depth. Demonstrate system-level behaviors exceeding individual component capabilities.
\end{functiondesc}

\begin{implementationbox}
\circuitimpl{Documented circuits: Induction (3--8 layers: previous-token $\rightarrow$ induction $\rightarrow$ MLP), IOI (8--12 layers: duplicate-token $\rightarrow$ S-inhibition $\rightarrow$ name-mover), Factual recall (5--15 layers: entity $\rightarrow$ MLP retrieval $\rightarrow$ output routing), Safety pipeline (E $\rightarrow$ F: detection $\rightarrow$ enforcement $\rightarrow$ refusal).}\\
\archimpl{Hierarchical depth organization: Early (0.00--0.25): surface processing (syntax, delimiters, detection). Middle (0.25--0.70): core computation (facts, entities, reasoning). Late (0.70--0.88): integration (routing, strategy, enforcement). Final (0.88--1.00): constraints (safety, formatting, completion).}\\
General pattern: detection/preparation (E) $\rightarrow$ core computation (M) $\rightarrow$ integration (L) $\rightarrow$ output (L-F).
\end{implementationbox}

\begin{ablationbox}
\textbf{Expected ablation:} Circuit-dependent effects. Induction circuit ablation: severe ICL loss. IOI circuit ablation: major entity routing failure. Factual circuit ablation: significant knowledge retrieval degradation. Safety pipeline ablation: critical safety failure. Circuit-specific rather than universal impact. Emergent capabilities lost when constituent mechanisms ablated.
\end{ablationbox}

\begin{examplebox}
\exinput{``Mary and John went to store. Mary gave book to...'' [IOI circuit]}\\
\exbehavior{Duplicate-token detects ``Mary'' repeat (M) $\rightarrow$ S-inhibition suppresses ``Mary'' (L) $\rightarrow$ name-mover outputs ``John'' (L)}\\
\exeffect{Correct indirect object through multi-stage circuit coordination}

\vspace{0.3em}
\textbf{Emergent capability - Few-shot learning:} Emerges from induction + entity tracking + output routing composition. Not explicitly trained but enables learning from 2--3 examples without parameter updates.
\end{examplebox}

\mechfooter{\statuswell}{pattern-completion, factual-recall, output-routing}

%-----------------------------------------------------------------------------
\subsubsection{Multi-Head Coordination Mechanism}
\label{mech:multi-head-coordination}

\noindent\depthinfo{All layers} | \primaryimpl{Architectural pattern} | \litnames{parallel processing, multi-aspect attention, head coordination}

\begin{functiondesc}
Enable parallel processing of multiple attention patterns within single layer. Allow different heads to focus on different relationships simultaneously: one head tracks entities, another tracks syntax, another handles facts. Combine multiple attention perspectives into unified representation. Support specialized head functions operating in parallel. Enable rich, multi-faceted information routing. Aggregate head contributions through linear combination and projection. Provide computational flexibility within layers through attention pattern diversity.
\end{functiondesc}

\begin{implementationbox}
\archimpl{Each attention layer contains multiple heads (8--64 depending on model). Each head independently computes attention: $\text{head}_i = \text{Attn}(Q_i, K_i, V_i)$. Outputs concatenated and projected: $\text{MultiHead} = \text{Proj}([\text{head}_1; \ldots; \text{head}_h])$.}\\
\circuitimpl{Parallel execution of diverse attention patterns. Heads specialize in different functions but contribute to shared residual stream. Enables simultaneous processing of multiple relationship types.}
\end{implementationbox}

\begin{ablationbox}
\textbf{Expected ablation:} Severe capability reduction. Model loses parallel processing power and specialization. Single attention pattern cannot handle multiple relationship types simultaneously. Major degradation across all tasks requiring multi-faceted attention. Reduced expressiveness and computational flexibility.
\end{ablationbox}

\begin{examplebox}
\exinput{``Alice told Bob about Paris while discussing travel''}\\
\exbehavior{Head 1: track entities (Alice, Bob, Paris); Head 2: extract facts (told, about); Head 3: maintain discourse (discussing travel); parallel execution}\\
\exeffect{Simultaneous processing of multiple relationships through head coordination}
\end{examplebox}

\mechfooter{\statuswell}{attention-mlp-composition, cross-layer-circuits}

%-----------------------------------------------------------------------------
\subsubsection{Attention-MLP Composition Logic Mechanism}
\label{mech:attention-mlp-composition}

\noindent\depthinfo{All layers} | \primaryimpl{Architectural pattern} | \litnames{interleaved processing, residual composition, layer coordination}

\begin{functiondesc}
Coordinate attention and MLP mechanisms through residual stream composition. Attention routes information, MLP transforms it, next attention routes transformed features. Enable information flow through alternating routing and transformation stages. Support incremental refinement: each layer adds to residual stream. Allow mechanisms to build on previous computations without overwriting. Implement universal approximation through composed operations. Core architectural pattern enabling complex computation from simple primitives.
\end{functiondesc}

\begin{implementationbox}
\archimpl{Fundamental transformer architecture: $h_{l+1} = h_l + \text{Attn}(h_l) + \text{MLP}(\text{Attn}(h_l) + h_l)$ where residual connections enable composition. Each layer adds incremental contribution. Direct path from input to any layer through skip connections.}\\
\circuitimpl{Universal pattern across all layers. Attention provides routing, MLP provides transformation, residual stream accumulates contributions. Enables arbitrarily complex computation through depth. Layers learn incremental refinements rather than complete transformations.}
\end{implementationbox}

\begin{ablationbox}
\textbf{Expected ablation:} Catastrophic failure. Without residual composition, model cannot function. Loss of information flow between layers. Each layer would overwrite previous computation rather than refining it. Fundamental to transformer operation. Training impossible for deep networks without gradient highways.
\end{ablationbox}

\begin{examplebox}
\exinput{Complex query requiring multiple processing stages}\\
\exbehavior{Layer 1 routes information, Layer 2 transforms via MLP, Layer 3 routes transformed features, iteratively through depth}\\
\exeffect{Progressive refinement through composed attention and MLP operations}
\end{examplebox}

\mechfooter{\statuswell}{multi-head-coordination, nonlinear-composition}

%-----------------------------------------------------------------------------
\subsubsection{Representational Superposition Mechanism}
\label{mech:representational-superposition}

\noindent\depthinfo{All layers} | \primaryimpl{Representational strategy} | \litnames{feature packing, compressed representation, polysemanticity}

\begin{functiondesc}
Represent more features than available dimensions through sparse superposed encoding. Pack multiple sparse features into shared neural dimensions. Enable efficient representation: model represents 100K+ features in 4K--8K dimensional space. Individual neurons respond to multiple concepts (polysemanticity). Features stored as sparse linear combinations in activation space. Trade representational efficiency for interference between features managed through sparsity. Enable rich representation within dimensional constraints. Extractable through sparse autoencoders into monosemantic features.
\end{functiondesc}

\begin{implementationbox}
\mlpimpl{Neurons encode multiple features through superposition. Activation space contains far more features than dimensions. Feature interference managed through sparsity: features rarely co-occur, enabling interference-limited storage.}\\
\saeimpl{Sparse autoencoders (10--100$\times$ expansion) extract individual features from superposed representations. Learned overcomplete basis reveals hidden structure. Enables fine-grained interpretability of polysemantic neurons.}\\
\circuitimpl{Fundamental representational strategy enabling rich feature sets within fixed architecture. Explains polysemanticity observations across transformer models.}
\end{implementationbox}

\begin{ablationbox}
\textbf{Expected ablation:} Cannot directly ablate (representational property not mechanism). SAE extraction reveals structure but removing superposition would require architectural change. Fundamental to how transformers achieve capability within dimensional constraints. Models without superposition would require impractically large dimensions.
\end{ablationbox}

\begin{examplebox}
\exinput{[Single neuron responding to multiple unrelated concepts: ``Paris'', ``capital'', ``tower'']}\\
\exbehavior{Neuron participates in representing multiple sparse features through superposition, firing for different concepts in different contexts}\\
\exeffect{Efficient packing of features; polysemantic neuron behavior enabling rich representation}
\end{examplebox}

\mechfooter{\statuswell}{nonlinear-composition, abstract-concepts}

%-----------------------------------------------------------------------------
\subsubsection{Infrastructure Primitives Mechanism}
\label{mech:infrastructure-primitives}

\noindent\depthinfo{N/A (architectural)} | \primaryimpl{See Appendix A} | \litnames{architectural primitives, foundational mechanisms}

\begin{functiondesc}
Brief reference entry linking to Appendix A containing architectural primitives without direct computational semantics. These foundational elements (residual connections, layer normalization, attention masking, embeddings, attention computation, gradient flow, tokenization) are essential for understanding transformer operation but are better documented separately as infrastructure rather than computational mechanisms in the main taxonomy.
\end{functiondesc}

\begin{implementationbox}
See Appendix A for detailed descriptions of 8 infrastructure primitives:
\begin{enumerate}[itemsep=0pt]
\item Residual Connections
\item Layer Normalization
\item Attention Masking
\item Embedding and Unembedding
\item Attention Computation
\item Gradient Flow
\item Tokenization
\item Feature Normalization
\end{enumerate}
\end{implementationbox}

\begin{ablationbox}
\textbf{Expected ablation:} See individual infrastructure primitive descriptions in Appendix A. Generally catastrophic failures as these are foundational architectural requirements.
\end{ablationbox}

\begin{examplebox}
Refer to Appendix A for examples of each infrastructure primitive.
\end{examplebox}

\mechfooter{\statuswell}{attention-mlp-composition, multi-head-coordination}


%=============================================================================
% DISCUSSION
%=============================================================================

\section{Discussion}
\label{sec:discussion}

\subsection{Key Patterns}

\textbf{Component specialization:} Attention specializes in routing (15 attention-primary mechanisms), MLPs in storage and transformation (21 MLP-required mechanisms). Sophisticated behaviors require multi-component circuits spanning 5--30 layers.

\textbf{Polyfunctionality:} Components contribute to multiple mechanisms depending on context. Duplicate-token heads serve pattern completion, entity tracking, and output routing. This computational reuse enables efficiency but complicates attribution.

\textbf{Depth-based cascades:} Mechanisms compose across depth in systematic patterns:
\begin{itemize}[itemsep=0.1em]
\item Early→Middle: Surface features prepare semantic processing
\item Middle→Late: Retrieved information feeds integration and routing
\item Late→Final: Content generation meets constraint enforcement
\end{itemize}

\subsection{Relationship to Prior Work}

This taxonomy integrates findings from:
\begin{itemize}[itemsep=0.1em]
\item Attention head taxonomies~\cite{zheng2025attention}: incorporated as mechanism implementations
\item MLP memory frameworks~\cite{geva2021transformer,geva2023dissecting}: integrated into full retrieval circuits
\item Circuit analyses~\cite{wang2022interpretability}: generalized into mechanism compositions
\item SAE feature catalogs~\cite{bricken2023towards}: connected to mechanism components
\end{itemize}

The contribution is integration through mechanism-first organization.

\subsection{The MLP Characterization Gap}

This taxonomy reveals significant asymmetry in mechanistic understanding between attention and MLP components:

\textbf{Attention mechanisms} (15 total, including attention-primary mechanisms): Well-characterized through attention pattern analysis, head-specific ablation, and circuit tracing. Tier 1 mechanisms like induction and IOI circuits have detailed component-level understanding with traced information flow across layers.

\textbf{MLP mechanisms} (18 mechanisms involve MLPs, 12 MLP-primary): Characterized primarily through:
\begin{itemize}[itemsep=0.1em]
\item Behavioral observation and ablation at layer level
\item Theoretical understanding of MLP function (key-value memory framework~\cite{geva2021transformer})
\item Sparse autoencoder feature analysis~\cite{bricken2023towards}
\item Neuron activation studies and knowledge localization~\cite{meng2022locating}
\end{itemize}

However, MLP mechanisms lack the circuit-level tracing available for attention. We understand \textit{what} MLPs compute (factual recall, semantic integration, abstract concepts) primarily through behavioral inference and theoretical frameworks rather than traced information flow through specific neuron populations.

This gap explains why many MLP-involving mechanisms are Tier 2 (behavioral evidence) rather than Tier 1 (mechanistic understanding). Factual recall achieves Tier 1 status through extensive neuron-level studies, but most other MLP mechanisms remain at Tier 2. Future research should prioritize MLP circuit tracing to achieve parity with attention mechanism understanding.

\textbf{Promising directions:}
\begin{itemize}[itemsep=0.1em]
\item Gradient-based MLP circuit attribution techniques
\item Feature-level ablation using sparse autoencoder features as units
\item Cross-layer MLP composition analysis tracking feature flow
\item Systematic neuron population studies identifying functional clusters
\item Causal tracing of MLP contributions to specific behavioral outcomes
\end{itemize}

\subsection{Limitations}

\textbf{Incomplete coverage:} Many capabilities (strategic planning, creative generation, abstract reasoning) lack mechanistic understanding.

\textbf{Tier imbalance:} Only 33\% of mechanisms achieve Tier 1 (mechanistic understanding). The remaining 67\% rely primarily on behavioral evidence. This reflects the current state of interpretability research but indicates substantial work remains to achieve full mechanistic characterization across all computational functions.

\textbf{Independence criteria challenges:} Some included mechanisms may fail strict independence tests under future scrutiny. The four-criterion framework provides structure but borderline cases remain. For example, is Focused Attention truly independent of Relevance Filtering, or a refinement stage of the same mechanism? Further research may reveal that some Tier 2 mechanisms are compositional rather than independent.

\textbf{Empirical gaps:} Several Tier 2 mechanisms need stronger validation through systematic ablation studies and component identification. The behavioral evidence is consistent but mechanistic understanding incomplete.

\textbf{Architecture dependence:} Focus on decoder-only transformers; generalization to other architectures requires investigation.

\textbf{Dynamic activation:} Mechanisms activate conditionally based on context; static descriptions don't capture this context-dependent engagement.

\textbf{Mechanism interactions:} Interference patterns between mechanisms (safety vs. helpfulness, format vs. reasoning) remain undercharacterized.

\subsection{Future Directions}

\textbf{Automated detection:} Develop tools to identify mechanisms in new models using taxonomy as specification.

\textbf{Complete MLP characterization:} Systematic study of MLP functions beyond key-value memory.

\textbf{Circuit composition rules:} Formalize patterns governing how mechanisms compose into sophisticated behaviors.

\textbf{SAE-mechanism integration:} Map sparse autoencoder features to mechanism components systematically.

\textbf{Cross-architecture transfer:} Test mechanism generalization to encoders, encoder-decoders, and alternative architectures.

\textbf{Mechanism editing:} Use understanding to improve capabilities through targeted interventions.

%=============================================================================
% CONCLUSION
%=============================================================================

\section{Conclusion}
\label{sec:conclusion}

This taxonomy provides mechanistic interpretability with standardized vocabulary bridging research traditions, functional organization revealing computational architecture, explicit multi-component integration patterns, and formal independence criteria establishing rigorous standards for mechanism identification. The framework enables researchers to communicate about computational functions while maintaining precision about implementations and evidence quality.

The taxonomy catalogs 33 mechanisms (11 Tier 1, 22 Tier 2) meeting independence criteria, with 3 additional patterns relocated to an appendix for future investigation. This distribution honestly reflects the current state of the field: strong mechanistic understanding of core attention-based mechanisms, behavioral characterization of MLP and reasoning mechanisms, and substantial work remaining to achieve complete circuit-level understanding.

\textbf{Adoption guidelines:}
\begin{enumerate}[itemsep=0.2em]
\item Use canonical mechanism names with literature cross-references
\item Specify implementation details: components, depths, circuit patterns
\item Indicate evidence tier (Tier 1: Mechanistic, Tier 2: Behavioral)
\item Apply independence criteria when proposing new mechanisms
\item Map components to mechanisms they implement
\item Use relative depth notation for cross-architecture comparison
\end{enumerate}

The taxonomy represents progress toward complete mechanistic understanding: explaining transformer capabilities through mechanism composition from architectural primitives. As interpretability research continues, mechanism-first organization with rigorous independence criteria will integrate findings into cumulative mechanistic knowledge while maintaining appropriate epistemic humility about evidence limitations.

%=============================================================================
% APPENDIX
%=============================================================================
%=============================================================================
% APPENDIX A: EMERGING MECHANISMS UNDER INVESTIGATION
%=============================================================================

\clearpage
\appendix

\section{Emerging Mechanisms Under Investigation}
\label{appendix:emerging}

This appendix catalogs computational patterns that show preliminary evidence but do not yet meet the independence criteria (Section~\ref{sec:framework}) for inclusion in the main taxonomy. These patterns may be:

\begin{itemize}[itemsep=0.2em]
\item Emergent behaviors from documented mechanism composition
\item Patterns requiring stronger mechanistic characterization
\item Potentially valid mechanisms needing independent confirmation across models
\item Components of other mechanisms rather than independent computational functions
\end{itemize}

Researchers are encouraged to investigate these patterns and provide evidence for or against their status as independent mechanisms using the four-criterion framework: (1) distinct ablation signature, (2) separable implementation, (3) independent activation, (4) non-compositional nature.

%-----------------------------------------------------------------------------
\subsection{Memory Consolidation}
\label{app:memory-consolidation}

\textbf{Hypothesized Function:} Integrate multiple memory sources (facts, schemas, entities) into coherent knowledge representations for complex queries requiring synthesis across memory types.

\textbf{Current Evidence:}
\begin{itemize}[itemsep=0.1em]
\item Behavioral observations suggest multi-source integration for complex queries
\item Ablation at layer level shows some degradation on knowledge-intensive tasks
\item Queries like ``Describe Einstein's contributions to physics in the context of early 20th century science'' appear to require coordinated retrieval
\end{itemize}

\textbf{Independence Challenge - Fails Criterion 4 (Non-Compositional):}

This pattern appears to be emergent composition of three documented mechanisms:
\begin{itemize}[itemsep=0.1em]
\item \textbf{Factual Recall} (Section~\ref{mech:factual-recall}): Retrieves Einstein facts
\item \textbf{Schema Retrieval} (Section~\ref{mech:schema-retrieval}): Accesses ``early 20th century science'' temporal/contextual framework
\item \textbf{Entity Grounding} (Section~\ref{mech:entity-grounding}): Links Einstein entity across mentions
\end{itemize}

No unique implementing structure has been identified beyond the coordination of these three mechanisms. The ``integration'' appears to occur through standard attention-MLP composition rather than specialized circuitry.

\textbf{Research Needed:}
\begin{itemize}[itemsep=0.1em]
\item Demonstrate ablation effects not explained by ablating constituent mechanisms
\item Identify unique implementing components (attention heads or MLP neurons) not shared with Factual Recall, Schema Retrieval, or Entity Grounding
\item Show independent activation: can Memory Consolidation be suppressed while preserving the three constituent mechanisms?
\item Alternative hypothesis: This may represent general compositional capability rather than a specific mechanism
\end{itemize}

%-----------------------------------------------------------------------------
\subsection{Local Context Modeling}
\label{app:local-context-modeling}

\textbf{Hypothesized Function:} Process immediate local context (1--5 tokens) for instruction markers, syntactic structure, and local patterns. Distinct from Pattern Completion which operates at broader scale (10+ tokens).

\textbf{Current Evidence:}
\begin{itemize}[itemsep=0.1em]
\item Behavioral observations of instruction detection in early layers
\item Format-sensitive tasks appear to rely on early-layer processing
\item N-gram neurons encode local sequential patterns
\end{itemize}

\textbf{Independence Challenge - May Be Component of Pattern Completion:}

Several concerns about independence:
\begin{enumerate}[itemsep=0.1em]
\item \textbf{Scale distinction unclear:} Is processing 1--5 tokens fundamentally different from 10+ tokens, or just Pattern Completion at smaller scale?
\item \textbf{Shared implementation:} Uses same components (n-gram neurons, early attention heads) as Pattern Completion
\item \textbf{Separability unclear:} Cannot ablate ``local'' pattern detection without affecting broader pattern detection
\item \textbf{May be implementation detail:} Early-layer operation may be \textit{how} Pattern Completion works rather than a separate mechanism
\end{enumerate}

\textbf{Research Needed:}
\begin{itemize}[itemsep=0.1em]
\item Demonstrate that local context processing (1--5 tokens) can be ablated independently of broader pattern completion (10+ tokens)
\item Identify components that respond specifically to local context without participating in broader pattern detection
\item Show qualitatively different computation for local vs. broader scales
\item Alternative hypothesis: This may describe the early-layer implementation of Pattern Completion rather than an independent mechanism
\end{itemize}

%-----------------------------------------------------------------------------
\subsection{Structural Boundary Tracking}
\label{app:structural-boundary-tracking}

\textbf{Hypothesized Function:} Track structural boundaries (sentence, paragraph, section markers, formatting delimiters, nested structure levels) to enable structure-aware processing.

\textbf{Current Evidence:}
\begin{itemize}[itemsep=0.1em]
\item Weak behavioral evidence from format-sensitive tasks
\item Some attention heads appear to attend to delimiter tokens
\item Models show some sensitivity to document structure
\end{itemize}

\textbf{Independence Challenge - Insufficient Evidence and Potential Component Status:}

Multiple concerns:
\begin{enumerate}[itemsep=0.1em]
\item \textbf{Insufficient mechanistic evidence:} Primarily inferred from behavioral observations without clear component identification
\item \textbf{May be Position-Based Processing:} Structural boundaries may be tracked through positional encodings rather than specialized mechanism
\item \textbf{May be Format Enforcement component:} Boundary awareness may be an implementation detail of how Format Enforcement works
\item \textbf{Weak ablation signature:} No clear demonstration of what behaviors are uniquely lost when ``boundary tracking'' is removed
\end{enumerate}

\textbf{Research Needed:}
\begin{itemize}[itemsep=0.1em]
\item Identify specific components (attention heads or MLP neurons) dedicated to boundary detection
\item Demonstrate ablation effects distinct from Position-Based Processing and Format Enforcement
\item Show structure tracking operates independently of positional information
\item Clarify whether this describes a mechanism or an implementation detail of other mechanisms
\item Alternative hypothesis: This may be an aspect of Position-Based Processing applied to structural markers
\end{itemize}

%-----------------------------------------------------------------------------
\subsection{Repetition \& Cycle Recognition}
\label{app:repetition-cycle-recognition}

\textbf{Note:} This pattern was merged into Pattern Completion Mechanism (Section~\ref{mech:pattern-completion}) as an implementation detail rather than relegated to this appendix. We include brief notes here for completeness.

\textbf{Status:} Repetition detection via duplicate-token heads serves multiple downstream mechanisms (pattern completion, entity tracking, output routing). However, this multi-purpose functionality represents component reuse rather than an independent mechanism. Duplicate-token heads are now documented as part of the Pattern Completion circuit implementation.

\textbf{Rationale for merger:} Repetition detection doesn't meet Criterion 2 (Separable Implementation) - the duplicate-token heads are shared across Pattern Completion, Entity Grounding, and Output Routing. This is computational reuse of components rather than an independent mechanism.

%-----------------------------------------------------------------------------
\subsection{Guidelines for Future Mechanism Proposals}

When proposing new mechanisms, researchers should:

\textbf{1. Apply Independence Criteria:}
\begin{itemize}[itemsep=0.1em]
\item Demonstrate distinct ablation signature not explained by other mechanisms
\item Identify separable implementing components with some unique structure
\item Show independent activation through targeted interventions
\item Prove non-compositional: not just emergent behavior from documented mechanisms
\end{itemize}

\textbf{2. Provide Strong Evidence:}
\begin{itemize}[itemsep=0.1em]
\item Systematic ablation studies across multiple models
\item Component-level identification (specific attention heads, MLP neurons, or circuits)
\item Reproducible demonstrations using standard techniques
\item Clear behavioral predictions that distinguish from similar mechanisms
\end{itemize}

\textbf{3. Consider Alternative Explanations:}
\begin{itemize}[itemsep=0.1em]
\item Could this be component reuse rather than a new mechanism?
\item Could this be composition of existing mechanisms?
\item Could this be an implementation detail of a broader mechanism?
\item Could this be explained by architectural primitives (position encoding, attention masking)?
\end{itemize}

\textbf{4. Document Limitations:}
\begin{itemize}[itemsep=0.1em]
\item Acknowledge evidence gaps and alternative interpretations
\item Specify which independence criteria are well-supported vs. uncertain
\item Identify research needed to definitively establish independence
\end{itemize}

The goal is cumulative mechanistic knowledge: each validated mechanism should advance understanding rather than multiply terminology for the same computational functions.


%=============================================================================
% BIBLIOGRAPHY
%=============================================================================
\clearpage
\bibliographystyle{plainnat}
\bibliography{bibliography}

\end{document}
