%=============================================================================
\subsection{Memory Recall Stack}
\label{sec:memory-stack}

\textbf{Stack overview:} Retrieve factual information, entity properties, and structured knowledge from model parameters. Move relevant information to output positions and suppress irrelevant content. Enable factual grounding and knowledge-based reasoning.

%-----------------------------------------------------------------------------
\subsubsection{Factual Association Retrieval}
\label{mech:factual-retrieval}

\noindent\depthinfo{0.35--0.75} | \primaryimpl{MLP neurons + Attention heads} | \litnames{fact retrieval, knowledge recall, factual memory}

\begin{functiondesc}
Retrieve factual associations and relationships stored in model parameters during training. Access learned knowledge: entity properties, relational facts, world knowledge. Implement distributed key-value memory: query patterns (keys) activate factual content (values). Store knowledge hierarchically across depths: surface patterns (early), core facts (middle), abstract concepts (late). Enable factual grounding without external retrieval.
\end{functiondesc}

\begin{implementationbox}
\mlpimpl{Knowledge neurons (M-L, 0.35--0.75) store factual associations distributed across layers. First sublayer: detect entity/query patterns (key matching). Second sublayer: provide factual content (value retrieval). Single fact distributed across multiple neurons; single neuron contributes to multiple facts.}\\
\attnimpl{Fact retrieval heads (M, 0.38--0.62) and entity heads (M, 0.35--0.65) identify factual queries and relevant entities for retrieval.}\\
\circuitimpl{Entity detection $\rightarrow$ MLP fact retrieval $\rightarrow$ name-mover output across 5--15 layers.}
\end{implementationbox}

\begin{ablationbox}
\textbf{Expected ablation:} Severe loss of factual knowledge. Linguistic fluency maintained but factual grounding lost. Major degradation on knowledge-intensive tasks and question answering. Model produces plausible-sounding but factually incorrect content.
\end{ablationbox}

\begin{examplebox}
\exinput{``The Eiffel Tower is located in...''}\\
\exbehavior{Detect ``Eiffel Tower'' entity, retrieve location association from parameters}\\
\exeffect{Output ``Paris'' via stored factual knowledge}
\end{examplebox}

\mechfooter{\statuswell}{entity-tracking, schema-retrieval, output-routing}

%-----------------------------------------------------------------------------
\subsubsection{Entity Tracking}
\label{mech:entity-tracking}

\noindent\depthinfo{0.30--0.65} | \primaryimpl{Attention heads} | \litnames{entity identification, named entity processing, entity linking}

\begin{functiondesc}
Identify, track, and link entity mentions across context. Recognize named entities (people, places, organizations) and their properties. Link different references to same entity: full names, abbreviations, nicknames, pronouns. Maintain entity representations across long contexts. Enable entity-aware processing for factual retrieval and reasoning. Ground references in specific entities rather than generic concepts.
\end{functiondesc}

\begin{implementationbox}
\attnimpl{Entity heads (M, 0.35--0.65) attend strongly to proper nouns and entity mentions. Duplicate-token heads (M, 0.30--0.58) detect repeated entity mentions. Coreference heads (M, 0.35--0.60) link different forms of same entity.}\\
\circuitimpl{Entity detection $\rightarrow$ factual retrieval $\rightarrow$ output routing. Parallel tracking of multiple entities across context.}
\end{implementationbox}

\begin{ablationbox}
\textbf{Expected ablation:} Significant degradation in entity-based reasoning and factual accuracy. Loss of entity linking and tracking. Major accuracy drop on who/what/where questions. Confusion between different entities with similar names.
\end{ablationbox}

\begin{examplebox}
\exinput{``Apple Inc. released new products. AAPL stock rose. The company...''}\\
\exbehavior{Link ``Apple Inc.'', ``AAPL'', and ``the company'' to single entity}\\
\exeffect{Maintain unified entity representation across mentions}
\end{examplebox}

\mechfooter{\statuswell}{factual-retrieval, coreference-resolution, output-routing}

%-----------------------------------------------------------------------------
\subsubsection{Schema Retrieval}
\label{mech:schema-retrieval}

\noindent\depthinfo{0.45--0.70} | \primaryimpl{MLP neurons + Attention heads} | \litnames{template retrieval, structural knowledge, procedural memory}

\begin{functiondesc}
Retrieve structured knowledge schemas, templates, and typical sequences from training. Access organizational patterns: event scripts (restaurant visit: enter, order, eat, pay, leave), document structures (paper: abstract, intro, methods, results), procedural knowledge (scientific method steps). Enable structured generation following learned patterns. Support script-based reasoning about typical situations and conventional formats.
\end{functiondesc}

\begin{implementationbox}
\mlpimpl{Schema storage neurons (M-L, 0.45--0.70) encode structured templates and procedural knowledge as hierarchical patterns.}\\
\attnimpl{Schema retrieval heads (M, 0.45--0.68) detect schema-triggering contexts and activate appropriate templates.}\\
\circuitimpl{Context detection $\rightarrow$ schema activation $\rightarrow$ structured generation.}
\end{implementationbox}

\begin{ablationbox}
\textbf{Expected ablation:} Loss of structured knowledge organization. Facts provided but poorly organized. Notable degradation on tasks requiring conventional formats or typical sequences. Reduced ability to follow procedural patterns.
\end{ablationbox}

\begin{examplebox}
\exinput{``Describe the water cycle.''}\\
\exbehavior{Retrieve cyclical process schema: evaporation $\rightarrow$ condensation $\rightarrow$ precipitation $\rightarrow$ collection}\\
\exeffect{Organized response following natural process structure}
\end{examplebox}

\mechfooter{\statusobs}{factual-retrieval, reasoning-integration}

%-----------------------------------------------------------------------------
\subsubsection{Output Routing}
\label{mech:output-routing}

\noindent\depthinfo{0.60--0.85} | \primaryimpl{Attention circuit} | \litnames{information movement, content copying, answer extraction}

\begin{functiondesc}
Move retrieved information to output positions where needed for generation. Route entities, facts, and content from earlier context to prediction position. Implement competitive selection among multiple candidates. Suppress incorrect alternatives while promoting correct content. Central mechanism for question answering, completion, and factual generation. Multi-stage process: candidate identification, competition, selection, movement.
\end{functiondesc}

\begin{implementationbox}
\circuitimpl{Name-mover heads (L, 0.60--0.80) attend to relevant content and copy to output. S-inhibition heads (L, 0.62--0.82) suppress contextually inappropriate alternatives. Copy-suppression heads (L, 0.65--0.85) prevent inappropriate repetition. Multi-head circuit implementing competitive selection.}\\
\attnimpl{IOI (Indirect Object Identification) circuit extensively studied: duplicate-token detection $\rightarrow$ S-inhibition $\rightarrow$ name-mover across 8--12 layers.}
\end{implementationbox}

\begin{ablationbox}
\textbf{Expected ablation:} Severe degradation in converting knowledge to output. Model knows facts but cannot output them correctly. Major accuracy drop on question answering and cloze completion. Entity confusion and selection errors in ambiguous contexts.
\end{ablationbox}

\begin{examplebox}
\exinput{``Alice and Bob went shopping. Alice gave the receipt to...''}\\
\exbehavior{Identify ``Bob'' as indirect object (not ``Alice''), suppress subject, move to output}\\
\exeffect{Complete with ``Bob'' (IOI circuit)}
\end{examplebox}

\mechfooter{\statuswell}{entity-tracking, factual-retrieval, S-inhibition}

%-----------------------------------------------------------------------------
\subsubsection{Coreference Resolution}
\label{mech:coreference}

\noindent\depthinfo{0.08--0.60} | \primaryimpl{Attention heads (multi-stage)} | \litnames{reference resolution, pronoun resolution, anaphora resolution}

\begin{functiondesc}
Resolve references to determine when different expressions refer to same entity. Handle pronouns, definite descriptions, demonstratives, possessives. Early-stage processing uses syntactic cues (gender, number, recency). Middle-stage processing integrates semantic understanding for ambiguous cases. Enable unified entity representations across diverse referring expressions. Critical for maintaining coherent entity tracking.
\end{functiondesc}

\begin{implementationbox}
\attnimpl{Reference resolution heads (E, 0.08--0.25) perform initial pronoun-to-noun binding using syntactic cues. Coreference heads (M, 0.35--0.60) resolve complex cases requiring semantic understanding.}\\
\circuitimpl{Multi-stage resolution: syntactic binding (E) $\rightarrow$ semantic integration (M) $\rightarrow$ entity tracking (M-L).}
\end{implementationbox}

\begin{ablationbox}
\textbf{Expected ablation:} Moderate to significant degradation in reference resolution. Increased errors on pronoun binding and entity tracking. Notable accuracy drop on reading comprehension requiring coreference. Later stages partially compensate for early stage loss.
\end{ablationbox}

\begin{examplebox}
\exinput{``The CEO announced changes. The executive clarified the policy. She emphasized...''}\\
\exbehavior{Link ``CEO'', ``executive'', and ``she'' to same entity across sentences}\\
\exeffect{Maintain consistent entity through different descriptions}
\end{examplebox}

\mechfooter{\statuswell}{entity-tracking, long-range-dependency}

%-----------------------------------------------------------------------------
\subsubsection{S-Inhibition}
\label{mech:s-inhibition}

\noindent\depthinfo{0.62--0.82} | \primaryimpl{Attention heads} | \litnames{negative selection, suppression, entity blocking}

\begin{functiondesc}
Suppress incorrect or contextually inappropriate content from being output. Named from IOI research: inhibit subject (S) when indirect object (IO) should be output. Work antagonistically with name-mover mechanisms to implement competitive selection. Prevent recently mentioned but incorrect entities from generation. Enable disambiguation when multiple candidates available. Implement negative selection by attending to wrong answers and decreasing their logits.
\end{functiondesc}

\begin{implementationbox}
\attnimpl{S-inhibition heads (L, 0.62--0.82) attend to entities that should NOT be output and apply negative contributions to their logits.}\\
\circuitimpl{Core component of IOI circuit and other disambiguation circuits. Works in opposition to name-mover heads to implement competitive selection.}
\end{implementationbox}

\begin{ablationbox}
\textbf{Expected ablation:} Moderate entity confusion and incorrect selections. Model outputs recently mentioned but contextually wrong entities. Notable accuracy loss in ambiguous contexts requiring entity disambiguation. Increased selection of subjects when objects should be output.
\end{ablationbox}

\begin{examplebox}
\exinput{``Alice gave the book to Bob. Then Alice...''}\\
\exbehavior{Inhibit ``Bob'' from output position after ``Alice'' to prevent incorrect continuation}\\
\exeffect{Enable correct continuation about Alice's actions}
\end{examplebox}

\mechfooter{\statuswell}{output-routing, copy-suppression, duplicate-token}
